\section{Avbrytelser}
\label{chp: avbrytelser} 

\subsubsection{Avbruddshåndtering}
Grandhi og Jones skisserer fire teknikker for avbruddshåndtering; forebygging, fraråding, modifiserte varslinger og forhåndsvisning. De argumenterer for at det ikke holder å kun vurdere avbruddets påvirkning på individets kognitive og sosiale kontekst, men at relasjonell kontekst også må inkluderes i “avbruddskonteksten”. Relasjonell kontekst defineres som alle aspekter mellom den som avbryter og den som blir avbrutt, hvilken relasjon de har, hva avbruddet dreier seg om og deres tidligere interaksjonsmønstre \cite{Grandhi10}. Grundgeiger og Sanderson skiller mellom «gode» og «dårlige» avbrudd, og hevder disse bør sees i sammenheng med hvilke effekter de har. Evjemo og Klemets kaller dette dualiteten ved avbrudd, personen som forårsaker avbruddet opplever å få en umiddelbar bekreftelse på at informasjonen er mottatt og kan dermed avlaste arbeidsminnet, mens den som blir avbrutt kan oppleve en negativ effekt, eksempelvis at den kognitive kapasiteten overskrides, forsinkelse i eget arbeid, stress og frustrasjon. Samtidig kan avbruddet ha en positiv effekt dersom den som blir avbrutt mottar en alarm om en pasients alvorlige tilstand, eller annen ønsket informasjon \cite{Evjemo, Grundgeiger09}.
