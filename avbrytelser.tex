\section{Avbrytelser}
\label{chp: avbrytelser} 

\subsubsection{Kapasitet}
“Stacking” defineres av Ebright som den usynlige beslutningsprosessen sykepleiere utfører om hva, hvordan og når de skal gi pleie til en tildelt gruppe pasienter. Observasjon av omgivelsene og stadig ny informasjon resulterer i en kontinuerlig re-prioritisering av hvilke oppgaver man skal gjøre når. Disse kognitive skiftene, kombinert med avbrytelser fra omgivelsene, kan resultere i en kognitiv belastning som kan hemme oppmerksomheten \cite{Ebright10}. Parker og Coiera hevder at en begrensende faktor i enhver kommunikasjonsanalyse er den kognitive kapasiteten individer har til å gjennomføre sitt arbeid, da studier har vist at feil og ineffektivitet er et resultat av at denne kapasiteten overskrides. Kunnskap om menneskets hukommelse hevdes å være nøkkelen til å forstå hvilke krav som bør settes til teknologi brukt i slike omgivelser \cite{Parker00}.
Det skilles normalt mellom langtids- og korttidsminne. Den passive kunnskapen man besitter ligger i langtidsminnet, for eksempel medisinske fakta eller viktige datoer, mens kortidsminnet, eller arbeidsminnet, er den bevisste delen av minnet som aktivt behandler informasjon. Arbeidsminnet har begrenset kapasitet og varighet, og lar seg raskt forstyrre av distraksjoner og avbrytelser. Coiera og Tombs antyder at synkron kommunikasjon foretrekkes fordi det gir en umiddelbar bekreftelse på at en beskjed er mottatt. Dersom man ønsker å gi en beskjed eller et ansvar videre, vil usikkerheten om beskjeden er mottatt bli liggende i arbeidsminnet frem til man får en bekreftelse fra mottaker \cite{Parker00}. 
Distribuert kognisjon, som presentert av Randall et al., handler om hvordan kognisjon er distribuert mellom individer i en gruppe, deres verktøy og omgivelser, og dermed hvordan kognitive gjenstander kan være til støtte for samarbeid. Disse gjenstandene kan være private ved at de gir informasjon til en enkelt bruker, eller de kan være offentlige og dermed gjøre informasjon synkront tilgjengelig for en samlokalisert gruppe.  

\subsubsection{Dualiteten ved avbrudd}
Grundgeiger og Sanderson skiller mellom «gode» og «dårlige» avbrudd, og hevder disse bør sees i sammenheng med hvilke effekter de har. Evjemo og Klemets kaller dette dualiteten ved avbrudd, personen som forårsaker avbruddet opplever å få en umiddelbar bekreftelse på at informasjonen er mottatt og kan dermed avlaste arbeidsminnet, mens den som blir avbrutt kan oppleve en negativ effekt, eksempelvis at den kognitive kapasiteten overskrides, forsinkelse i eget arbeid, stress og frustrasjon. Samtidig kan avbruddet ha en positiv effekt dersom den som blir avbrutt mottar en alarm om en pasients alvorlige tilstand, eller annen ønsket informasjon \cite{Evjemo, Grundgeiger09}. 

\subsubsection{Avbruddshåndtering}
Grandhi og Jones skisserer fire teknikker for avbruddshåndtering; forebygging, fraråding, modifiserte varslinger og forhåndsvisning. De argumenterer for at det ikke holder å kun vurdere avbruddets påvirkning på individets kognitive og sosiale kontekst, men at relasjonell kontekst også må inkluderes i “avbruddskonteksten”. Relasjonell kontekst defineres som alle aspekter mellom den som avbryter og den som blir avbrutt, hvilken relasjon de har, hva avbruddet dreier seg om og deres tidligere interaksjonsmønstre \cite{Grandhi10}.
