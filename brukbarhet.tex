\subsection{Brukbarhet}
\label{chp: brukbarhet}

Brukbarhet defineres i ISO 9241-11 \cite{Svanes08} som

\noindent
\begin{otherlanguage}{english}
\emph{the extent to which a product can be used by specified users to achieve specified goals with effectiveness, efficiency and satisfaction in a specified context of use.}
\end{otherlanguage}

\noindent
Brukbarhet er dermed et begrep som ikke kan måles generelt, men er relativt til bestemte brukere med bestemte mål i en spesifikk kontekst. Likevel kan man generelt definere noe som brukbart dersom det er funksjonelt, effektivt og tilfredsstillende \cite{Kuniavsky}. Et system er funksjonelt når det anses som nyttig av brukeren, og må dermed svare til de forventningene brukeren har til hva det skal gjøre, og faktisk gjøre det. Effektivitet kan måles som hvor raskt en er i stand til å utføre en oppgave med så lite feil som mulig. Det er vanskelig å konkretisere hva som gjør et system tilfredsstillende for en bruker, dette er individuelt og er en følelsesmessig respons ved bruk systemet \cite{Kuniavsky}. Ofte vil god design være avgjørende for om en bruker opplever systemet på en god måte, forutsett at det også er funksjonelt og effektivt. Designere av grensesnitt har gjennom årene kommet frem til en rekke reningslinjer for god design. Desverre er disse ofte blitt kritisert for å være både for spesifikke og ufullstendige \cite{mmi}. 
Interaktive applikasjoner og systemer vil spesielt hemmes av dårlig design dersom de allerede er vanskelige å lære og kompliserte å bruke. Slike systemer kan også føre til katastrofale utfall dersom kritisk informasjon ikke blir presentert på en effektiv måte. Det er derfor avgjørende å ha forståelse for hva slags informasjon brukeren trenger og hvordan denne skal presenteres \cite{Ebright10}. 

\subsubsection{Gestaltprinsippene}
Gestaltprinsippene har sitt navn fra det tyske \emph{gestalten}, som betyr "å forme". Prinsippene blir ofte referert til som lover, og det finnes mange varianter utviklet av forskjellige psykologer. Det de har til felles at de forklarer hvordan vi organiserer og tolker visuelle intrykk i områder og strukturer. Disse prinsippene beskriver blant annet virkningen av plassering av fokuspunkt, organisering av elementer, fargevalg, harmoni og enkelhet, og understeker at vi alle tolker visuelle inntrykk ut ifra egne erfaringer. 
Disse prinsippene ble i utgangspunktet brukt til å foreslå hvordan statiske visuelle elementer burde presenteres. I dag er det vanlig å ta hensyn til disse retningslinjene i design av blant annet skjermbilder \cite{Chang02}. 

\subsubsection{Affordance, konseptuelle modeller og begrensninger}
Vår forståelse av hvordan vi skal bruke en gjenstand første gang vi ser den beror på tre dimensjoner: konseptuell modell, begrensninger og \emph{affordance}. Ordet affordance ble innført av psykologen J. J. Gibson. Det finnes ingen norsk oversettelse, men begrepet refererer til hva slags handling en gjenstand signaliserer når du ser den for første gang \cite{Norman99}.

\noindent
Vi skiller mellom ekte affordance og oppfattet affordance, hvor det først og fremst er oppfattet affordance vi kan kontrollere i skjermdesign. Ekte affordance vil med tanke på datamaskiner være tastatur, skjerm, knapper og musepeker som signaliserer handlinger som berøring, peking, trykking og å se på. Ekte affordance finnes alstå uavhengig av hva som vises på skjermen. Det som vises på skjermen er visuelle tilbakemeldinger som viser hva vi kan gjøre, altså oppfattet affordance \cite{Norman99}.

\noindent
Affordance blir ofte forvekslet med begrensninger og konvensjoner. Vi kan si å ha tre typer begrensninger:
\begin{enumerate}
\item Fysiske begrensninger, som har sammenheng med ekte affordance. Et eksempel på dette kan være at det er ikke mulig å flytte musepekeren utenfor skjermen. 
\item Logiske begrensninger er sterkt knyttet til en god konseptuell modell. Et eksempel på dette er hvordan brukeren vet at den må skrolle for å se resten av siden.
\item Kulturelle begrensninger er tillært og deles av en gruppe mennesker, eksempelvis hva det betyr når markøren skifter form.
\end{enumerate}
En konvensjon er en kulturell begrensning som har utviklet seg over tid, og som forbyr noe og oppfordrer til noe annet. Slike konvensjoner er langsomme, i den betydning at det tar lang tid før de blir adoptert, og når de først er blitt det, tar det vel så lang tid før de forsvinner \cite{Norman99}.
\noindent
Spesielt logiske og kulturelle begrensninger er sterke virkemidler innen skjermdesign, da designere er mer opptatt av hva brukerene oppfatter som mulig, fremfor hva som faktisk er sant. Tilbakemeldinger og reaksjoner fra skjermen hjelper oss å forstå hva vi kan og skal gjøre \cite{Norman99}.