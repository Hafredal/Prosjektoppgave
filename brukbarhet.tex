\subsection{Brukbarhet}
\label{chp: brukbarhet}

Brukbarhet defineres i ISO 9241-11 \cite{Svanes08} som

\noindent
\begin{otherlanguage}{english}
\emph{the extent to which a product can be used by specified users to achieve specified goals with effectiveness, efficiency and satisfaction in a specified context of use.}
\end{otherlanguage}

\noindent
Brukbarhet er dermed et begrep som ikke kan måles generelt, men er relativt til bestemte brukere med bestemte mål i en spesifikk kontekst. Likevel kan man generelt definere noe som brukbart dersom det er funksjonelt, effektivt og tilfredsstillende \cite{Kuniavsky}. Et system er funksjonelt når det anses som nyttig av brukeren, og må dermed svare til de forventningene brukeren har til hva det skal gjøre, og faktisk gjøre det. Effektivitet kan måles som hvor raskt en er i stand til å utføre en oppgave med så lite feil som mulig. Det er vanskelig å konkretisere hva som gjør et system tilfredsstillende for en bruker, dette er individuelt og er en følelsesmessig respons ved bruk systemet \cite{Kuniavsky}. Ofte vil god design være avgjørende for om en bruker opplever systemet på en god måte, forutsett at det også er funksjonelt og effektivt. Designere av grensesnitt har gjennom årene kommet frem til en rekke reningslinjer for god design. Desverre er disse ofte blitt kritisert for å være både for spesifikke og ufullstendige \cite{mmi}. 
Interaktive applikasjoner og systemer vil spesielt hemmes av dårlig design dersom de allerede er vanskelige å lære og kompliserte å bruke. Slike systemer kan også føre til katastrofale utfall dersom kritisk informasjon ikke blir presentert på en effektiv måte. Det er derfor avgjørende å ha forståelse for hva slags informasjon brukeren trenger og hvordan denne skal presenteres \cite{Ebright10}. 