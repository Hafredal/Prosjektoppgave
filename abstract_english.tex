\pagestyle{empty}
\begin{abstract}

\noindent
A hospital is an organization demanding efficiency an communication to ensure good patient care and safety. For the nurses, information concerning the activity of colleagues can be essential for their coordination and prioritization of work. Previous research suggests that this information often is communicated orally, but that there may be situations where this kind of communication is challenging. Flow of information, and external interruptions causes the nurses to continuously re-prioritize their activities.

\noindent
This project assignment deals with the nurse call system at St. Olavs Hospital in Trondheim. We have explored the current work practices to detect whether information about colleagues' activities and their availability is useful for the nurses in their prioritization and organization of work, and
how such information can be distributed appropriately.  Based on the theory saying that external interruptions may affect individuals' cognitive capacity, and potentially have negative effects, we have considered whether it is possible to reduce such effects. Through studies of previous work we have tried to identify the benefits and challenges regarding the system used today. Based on this we have made a prototype suggesting improvements. Using workshops as a methodical approach
to answer the research questions, the prototype have been tested and evaluated by a group of nursing students. We will in this paper present the results form the workshops, and discuss whether information about colleagues' activities and availability is useful, and how such information should be distributed.


%\Blindtext[5][1]
\end{abstract}