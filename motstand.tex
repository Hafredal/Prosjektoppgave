\subsection{Motsand mot endring}
\label{chp: Motstand}

En annen utfordring er når de ansatte viser motsand mot den endringen implementasjonen av et nytt informasjonssystem må sies å være. en slik endring vil påvirke mennsekene som jobber der, deres sosiale relasjoner og forholdet mellom mennesker i og utenfor organisasjonen. Derfor kan, og vil det oppstå mer eller midre motstand mot endringene, gjerne uavhengig av hvor "godt" systemet som implementeres er, og i hvor stor grad de ansatte har fått bidra i form av medvirkning. Det finnes mange årsaker til slik motstand mot endring. Vi vil her forklare nærmere noen av dem, som vi mener er relevante for oppgaven. 

\subsubsection{Faglig uenighet}
De ansatte kan altså være faglig uenige i selve endringen. Hverken analyser av dagens situasjon, behovet for endring eller selve endringen er klare og objektive størrelser \cite{Jacobsen12}, og det er derfor også rom for uenigheter rundt det faktiske behovet for endring, og hvorvidt endringen som gjennomføres er den rette løsningen på problemet. 

\subsubsection{Ekstraarbeid}
En endring i seg selv kan kreve ekstraarbeid fra de ansatte, spesielt i en overgangsfase. Det kan være mye nytt å sette seg inn i og lære, noe som krever en eksra innsats, ofte uten tilstrekkelig kompensasjon, noe mange stiller seg negative til \cite{Jacobsen12}. Slikt ekstraarbeid kan i tillegg til å lære noe nytt innebære at man må avlære de gamle måtene å jobbe på. 