\subsection{Motsand mot endring}
\label{chp: Motstand}

Når de ansatte viser motsand mot den endringen implementasjonen av et nytt informasjonssystem må sies å være, kan dette by på utfordringer, både for ledelsen og utviklere. En slik endring vil påvirke mennsekene som jobber der, deres sosiale relasjoner og forholdet mellom mennesker i og utenfor organisasjonen. Derfor kan, og vil det oppstå motstand mot endringene, gjerne uavhengig av hvor "godt" systemet som implementeres er, og i hvor stor grad de ansatte har fått bidra i form av medvirkning \cite{Jacobsen12}. Cavaye (1995) kaller denne motstanden for mellomliggende variable, og understreker at slike hinder kan være fremtredende selv om brukerene både har et godt forhold med designer/utvikler og deres innspill blir hørt. Det finnes flere årsaker til slik motstand mot endring, og vi vil her forklare noen av dem, som vi ser på som relevante for denne oppgaven. 

\subsubsection{Faglig uenighet}
De ansatte kan være faglig uenige i selve endringen. Verken analyser av dagens situasjon, behovet for endring eller selve endringen er klare og objektive størrelser, og det er derfor også rom for uenigheter rundt det faktiske behovet for endring, og hvorvidt endringen som gjennomføres er den rette løsningen på problemet. \cite{Jacobsen12}

\subsubsection{Ekstraarbeid}
En endring i seg selv kan kreve ekstraarbeid fra de ansatte, spesielt i en overgangsfase. Det kan være mye nytt å sette seg inn i og lære, noe som krever en ekstra innsats, ofte uten tilstrekkelig kompensasjon, noe mange stiller seg negative til. Slikt ekstraarbeid kan i tillegg til å lære noe nytt innebære at man må avlære de gamle måtene å jobbe på. \cite{Jacobsen12}

\subsubsection{Systemet som en trussel}
Dersom brukerene ser på det nye systemet som en trussel mot deres nåværende kontroll over eget arbeid, eller deres posisjon i form av deres ekspertise, vil de mest sannsynlig motsette seg endringen det nye systement representerer. \cite{Cavaye95}

\subsubsection{Grad av medvirkning}
Motstand kan også oppstå dersom det faktiske nivået av medvirkning avviker fra det nivået brukeren ønsker seg. Dette gjelder ikke bare dersom nivået er lavere, men også dersom nivået av medvirkning blir høyere enn det brukeren så for seg i utgangspunktet. Det er derfor ikke tilstrekkelig med medvirkning i seg selv, den må også møte brukerenes forventninger. \cite{Cavaye95}

\noindent
Jacobsen(2012) hevder at indre motivasjon og involvering de ansatte, slik at de føler seg som medeiere i endringsprosessen er avgjørende for å få de ansatte motivert for endringen og dermed redusere overnevnte motstand. Bred deltagelse på denne måten gir den enkelte ansatte opplevelsen av at den er med på å forme sin egen fremtid, og at dette vil skape en aksept og forståelse for usikkerheten som er assosiert med en hver endring.