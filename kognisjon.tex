\section{Kognitiv kapasitet og distribusjon}
\label{chp: kognisjon}

\subsubsection{Kognitiv kapasitet}
“Stacking” defineres av Ebright (2010) som den usynlige beslutningsprosessen sykepleiere utfører om hva, hvordan og når de skal gi pleie til en tildelt gruppe pasienter. Stadige endringer i omgivelsene og informasjonsflyt resulterer i en kontinuerlig re-prioritisering av hvilke oppgaver man skal gjøre når. Disse kognitive skiftene, kombinert med avbrytelser fra omgivelsene, kan resultere i en kognitiv belastning som kan hemme oppmerksomheten \cite{Ebright10}. Parker og Coiera (2000) hevder at en begrensende faktor i enhver kommunikasjonsanalyse er den kognitive kapasiteten individer har til å gjennomføre sitt arbeid, da studier har vist at feil og ineffektivitet er et resultat av at denne kapasiteten overskrides. Kunnskap om menneskets hukommelse hevdes å være nøkkelen til å forstå hvilke krav som bør settes til teknologi brukt i slike omgivelser \cite{Parker00}. Det skilles normalt mellom langtids- og korttidsminne. Den passive kunnskapen man besitter ligger i langtidsminnet, for eksempel medisinske fakta eller viktige datoer, mens kortidsminnet, eller arbeidsminnet, er den bevisste delen av minnet som aktivt behandler informasjon. Arbeidsminnet har begrenset kapasitet og varighet, og lar seg raskt forstyrre av distraksjoner og avbrytelser. Coiera og Tombs antyder at synkron kommunikasjon, ansikt-til-ansikt eller per telefon, foretrekkes fordi det gir en umiddelbar bekreftelse på at en beskjed er mottatt. Dersom man ønsker å gi en beskjed eller et ansvar videre, vil usikkerheten om beskjeden er mottatt bli liggende i arbeidsminnet frem til man får en bekreftelse fra mottaker \cite{Parker00}. 

\subsubsection{Kognitiv distribusjon}
Distribuert kognisjon, som presentert av Randell et al., handler om hvordan kognisjon er distribuert mellom individer i en gruppe, deres verktøy og omgivelser, og dermed hvordan kognitive gjenstander kan være til støtte for samarbeid. Disse gjenstandene kan være private ved at de gir informasjon til en enkelt bruker, eller de kan være offentlige og dermed gjøre informasjon synkront tilgjengelig for en samlokalisert gruppe \cite{Randell}.  