\chapter{Konklusjon}
\label{chp:konklusjon}

Våre funn avdekker at sykepleierne bruker kunnskapen de har om kolleger og inneliggende pasienter når de vurderer hvordan de skal håndtere innkommende pasientsignaler. Spesielt bruker sykepleierne sin sosiale og romlige awareness, som i stor grad oppstår som et resultat av informasjonen som gis under vaktskiftemøtet. Dette resulterer også i distribuert kognisjon, da sykepleierne i fellesskap innehar awareness om mål, planer og detaljer. For god koordinasjon er awareness om individuelle og felles aktiviteter avgjørende, men da sykepleierne i varierende grad synliggjør sine aktiviteter i løpet av vakten, kan denne felles oversikten reduseres. Vi har sett at ved å automatisere sykepleiernes tilstedemarkering vil pålitelig informasjon om sykepleiernes aktiviteter, resultere i økt sosial awareness. Informasjon om kollegers aktiviteter og tilgjengelighet kan være nyttig i situasjoner hvor de må håndtere innkommende pasientsignaler, men hvorvidt denne informasjonen vil ha samme nytteverdi i andre situasjoner er ikke avklart.

\noindent
Da sykepleierne heller vil avbrytes enn å gå glipp av informasjon kan vi konkludere med at informasjonens verdi overgår avbruddenes negative effekter, men at de disse fortsatt kan reduseres på flere måter. Mer informasjon om pasienten som utløser signalet, eksempelvis et oversiktsbilde av pasientens rom, eller en kort beskrivelse av dennes tilstand, kan støtte sykepleiernes prioritering av oppgaver. Ved å implementere statuser for å vise sykepleieres tilgjengelighet kan man: (1) fraråde andre sykepleiere å avbryte disse, og (2) endre rekkefølgen på hvilke sykepleiere som blir varslet om signalet basert på deres status, fremfor bemanningsplanen.  Modifisering av signalets varsling, vil føre til at avbrytelsen forårsaket av signalet kan være mindre forstyrrende. Samtidig er det viktig at sykepleierne ikke går glipp av informasjon, og det kan derfor være hensiktsmessig å flytte rompanelene slik at sykepleieren kan oppdage varsling på disse, uten at det forstyrrer pasienten. 

\noindent
Da det er klart at det gjenstår mye forskning før et slikt system kan implementeres vil vi ved videre utvikling av prototypen anbefale å fokusere på funnene beskrevet over. I tillegg vil det i neste omgang være nyttig å teste systemet i mer realistiske omgivelser, med flere deltagere og mer komplekse senarioer.
