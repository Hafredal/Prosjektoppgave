\chapter{Konklusjon}
\label{chp:konklusjon}


Er informasjon om kollegers aktiviteter og tilgjengelighet nyttig for sykepleiere, og hvordan kan slik informasjon distribueres på en hensiktsmessig måte? 
Hvordan kan systemet endres for å begrense potensielle negative effekter ved avbrytelser?

Våre funn avdekker at sykepleierne bruker kunnskapen de har om kolleger og inneliggende pasienter når de vurderer hvordan de skal håndtere innkommende pasientsignaler. Spesielt bruker sykepleierne sin sosiale og romlige awareness, som i stor grad oppstår som et resultat av informasjonen som gis under vaktskiftemøtet. Dette resulterer også i distribuert kognisjon, da sykepleierne i fellesskap innehar awareness om mål, planer og detaljer. For god koordinasjon er awareness om individuelle og felles aktiviteter avgjørende, men da sykepleierne i varierende grad synliggjør sine aktiviteter i løpet av vakten, kan denne felles oversikten reduseres. Om dette er et problem kan ikke sies med sikkerhet, men vi har sett at informasjon om kollegers aktiviteter og tilgjengelighet kan være nyttig i situasjoner hvor de må håndtere innkommende pasientsignaler. Hvorvidt denne informasjonen vil ha samme nytteverdi i andre situasjoner er uklart.

\noindent
Kognitive artefakter, som skjermer, lister og timeplaner, er offentlige og distribuerer informasjon for å opprettholde oversikt over det totale arbeidet. Vi har i dette prosjektet evaluert om en applikasjon kan distribuere slik informasjon på en hensiktsmessig måte.  

\noindent
Under utspillingen av scenarioene hvor en sykepleier var inne hos en pasient, var det tydelig at noen av deltagerne opplevde negative effekter forårsaket av det innkommende pasientsignalet. Samtidig ville ingen av studentene være foruten informasjonen som ble varslet, da det ses på som en positiv effekt å motta perifer informasjon.
Vi fant likevel at det er behov for endringer av systemet, både med tanke på hvilken informasjon som formidles, og hvordan dette varsles. Slik systemet fungerer i dag blir både sykepleiere og pasienter forstyrret av signalet, og det vil dermed være hensiktsmessig å endre systemet på en slik måte at avbrytelsen ikke er like forstyrrende. Basert på funnene er lyden den største kilden til forstyrrelse, og en mulig løsning på dette er varsling med vibrasjon, eller lavere volum. Denne endringen kan resultere i at det er vanskeligere å oppdage det innkommende signalet. En mulig løsning på dette er å flytte rompanelene, da de i dag er plassert slik at sykepleierne ikke ser dem når de er i samtale med pasienten. 


Vi ser dermed at en innføring av statuser som formidler sykepleiernes tilgjengelighet, kan være en løsning på disse problemene.  

flere "grunnsteiner" for awareness: Bardrams tre, to typer redundans(ifølge Cabitza kan awareness oppstå pga dette). -> godt grunnlag