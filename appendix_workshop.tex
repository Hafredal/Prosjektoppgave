\chapter{Plan for Workshop}
\label{appendix_workshop}


\textbf{Dette gjøres på forhånd:}
\begin{itemize}
  \item Klargjøre laben og artefakter
  \item Lage navneskilt
\end{itemize}

%\begin{adjustbox}{with=\textwith, hight=\texthight}
\begin{table}[H]
%\small
\centering
\begin{tabular}{l|l|l|l|l}
\hline
\textbf{\begin{tabular}[x]{@{}c@{}}Del av\\workshop\end{tabular}} & \textbf{Beskrivelse} & \textbf{Hvor} & \textbf{Tidspunkt} & \textbf{Varighet}\\
\hline
Steg 1 & Informasjon & \begin{tabular}[x]{@{}c@{}}Rundt bord\\/i gangen\end{tabular} & 13:00-13:15 & 15min\\
\hline
Steg 2 & Fokusgruppe med scenarioer & \begin{tabular}[x]{@{}c@{}}Rundt bord\\/i gangen\end{tabular} & 13:15-13:45 & 30min\\
\hline
& \textbf{Pause} & & 13:45-13.50 & 5min\\
\hline
Steg 3 &\begin{tabular}[x]{@{}c@{}} Scenarioer med rollespill\\- uten prototype\end{tabular} & Inne på sengerom & 13:50-14:25 & 35min\\
\hline
Steg 4 &\begin{tabular}[x]{@{}c@{}} Scenarioer med rollespill\\- med prototype\end{tabular} & Inne på sengerom & 14:25-15:00 & 35min\\
\hline
& \textbf{Pause} & & 15:00-15:15 & 15min\\
\hline
Steg 5 & Fokusgruppe/oppsummering & \begin{tabular}[x]{@{}c@{}}Rundt bord\\/i gangen\end{tabular} & 15:15-16:00 & 45min
\end{tabular}
\caption{Overordnet plan for dagen}
\label{OverordnetPlan}
\end{table}
%\end{adjustbox}

\pagebreak

\textbf{Forskningsspørsmål:}
\begin{enumerate}
  \item Identifisere behov knyttet til funksjonalitet for støtte av distribusjon av awareness-informasjon.
  \item Hvordan kan systemet endres for å begrense potensielle negative effekter ved avbrytelser?
\end{enumerate}

\subsubsection{Steg1: Informasjon - 13:00-13:15 (15min)}
\begin{itemize}
  \item Ønsker velkommen
  \item Presentasjonsrunde
  \item Presentasjon av roller (fasilitator, pasienter etc\ldots)
  \item Presenterer planen for workshopen og dens hensikt
  \begin{itemize}
  \item Vil vil først ha en felles diskusjon her, hvor vi ønsker å avdekke dagens situasjon mtp ansvarsfordeling og organisering av arbeidet
  	\item Vi vil deretter spille ut et par scenarioer på sengerommene her, hvor dere bruker de telefonene og funksjonene dere er vant med
  	\item Deretter vil vi forklare funksjonaliteten til prototypen vi har laget, for så å spille ut et par scenarier hvor vi ønsker at dere skal bruke denne
  	\item Til sist ønsker vi en oppsumering/diskusjon rundt disse scenarioene og deres syn på prototypen
  \end{itemize}
\end{itemize}

\noindent
Vi ønsker at alle bidrar i alle diskusjonene. Vi er her fordi vi ønsker å høre deres tanker/meninger

\subsubsection{Steg 2: Fokusgruppe med scenarioer}

\begin{table}[H]
\small
\begin{tabular}{p{4cm}|p{5cm}|l|l}
\hline
\textbf{Scenario} & \textbf{Oppfølgingsspørsmål} & \textbf{Tidspunkt} & \textbf{Varighet}\\
\hline
\emph{Rollen som primærsykepleier:} Du har ansvar for to av pasientene på tunet. \textbf{Hvordan vil du definere/beskrive denne rollen?} & I hvilken grad kjenner du til de andre pasientene? \begin{itemize}
\item Hvem har ansvar?
\item sykdomsbilde og tilstand
\item lokasjon
\item hvem og hvilket type rom
\end{itemize}
& 13:15-13:25 & 10min\\
\hline
\emph{Rollen som disp:} Du er disp for to av pasientene på tunet. \textbf{Hvordan vil du definere/beskrive denne rollen?} & I hvilken grad påvirker det beslutningen du tar om å svare på et pasientsignal, når du får et anrop fra en pasient du er disp for?
 & 13:25-13:35 & 10min\\
\hline
\emph{Utilgjengelig:} Du havner i en situasjon hvor du vil være utilgjengelig i en periode.
\begin{itemize}
\item i hvilke situasjoner er det ønskelig å kunne være “utilgjengelig” for henvendelser?
\item Hvilke henvendelser vil du ønske å motta uansett situasjon?
\item Hvilke henvendelser vil du helst ikke motta? Hvorfor?
\end{itemize}
& 
\begin{itemize}
\item Hvem informeres (kolleger/pasienter)?
\item Hva med telefonen?
\item Endres bemanningsplanen?
\item Lunchpause?
\item sårskift?
\item Urolig psient?
\item Andre situasjoner?
\end{itemize}
& 13:35-13.45 & 10min\\
\end{tabular}
\caption{Scenarioer til fokusgruppe}
\label{Steg2}
\end{table}

\textbf{Pause - 5min}
\pagebreak
\subsubsection{Steg 3: Scenarioer med rollespill}



\begin{itemize}
\item Presentasjon av pasientenes sykdomshistorier rundt bordet før vi starter rollespillene
\end{itemize}

\begin{table}[H]
\small
\begin{tabular}{p{3cm}|p{2cm}|p{4cm}|l|l}
\hline
\textbf{Scenario} & \textbf{Aksjon} & \textbf{Oppfølgingsspørsmål} & \textbf{Tidspunkt} & \textbf{Varighet}\\
\hline
Du står i gangen, en annen sykepleier er også til stede. Du skal inn til Jonas fordi du vet han er veldig usikker og bekymret, og vil høre hvordan det går & & Hva slags forberedelser gjør du og hvorfor? (sier du fra til andre?)
& 13:50-13:55 & 5min\\
\hline
Du er inne hos Jonas og har en seriøs samtale om hans tilstand & Adam utløser pasientsignal. & 
\begin{itemize}
\item Hva gjør du?
\item Hvor ser du?
\item Ønsker du å fortsette samtalen uavbrutt eller forlater du rommet?
\item Hvordan påvirker signalet situasjonen?
\item Hvilken informasjon ville satt deg i bedre stand til å ta en avgjørelse?
\item Hvis du ikke velger å forlate rommet, hvorfor blir du?
\item Dersom du forlater rommet, og det viser seg at Adam ønsket et glass vann, føler du at du tok riktig avgjørelse?
\end{itemize}
 & 13:55-14:00 & 5min\\
\end{tabular}
\caption{Scenario 1 for pasientsignal uten prototype}
\label{Steg3.1}
\end{table}

\begin{table}[H]
\small
\begin{tabular}{p{3cm}|p{2cm}|p{4cm}|l|l}
\hline
\textbf{Scenario} & \textbf{Aksjon} & \textbf{Oppfølgingsspørsmål} & \textbf{Tidspunkt} & \textbf{Varighet}\\
\hline
Du står i gangen, en annen sykepleier er også til stede. Du skal inn til Adam for å utføre sårskift, og har kledd deg opp med stellefrakk  mm. & & Hva slags forberedelser gjør du og hvorfor? (sier du fra til andre?)
& 14:05-14:10 & 5min\\
\hline
Du er inne hos Adam og Utfører sårskift & Jonas utløser pasientsignal. & 
\begin{itemize}
\item Hva gjør du?
\item Hvor ser du?
\item Ønsker du å fortsette sårskiftet eller forlater du rommet? Hvilke faktorer spiller inn?
\item Hvordan påvirker signalet situasjonen?
\item Hvordan blir arbeidet påvirket?
\item Ville du satt deg som utilgjengelig? Sagt fra til noen om at du skal foreta et sårskift?
\item Dersom du forlater rommet, og det viser seg at Jonas ønsket et glass vann, føler du at du tok riktig avgjørelse?
\end{itemize}
 & 14:10-14:25 & 15min\\
\end{tabular}
\caption{Scenario 2 for pasientsignal uten prototype}
\label{Steg3.2}
\end{table}

\pagebreak

\begin{table}[H]
\small
\begin{tabular}{p{3cm}|p{2cm}|p{4cm}|l|l}
\hline
\textbf{Scenario} & \textbf{Aksjon} & \textbf{Oppfølgingsspørsmål} & \textbf{Tidspunkt} & \textbf{Varighet}\\
\hline
Du er inne hos Jonas og har en seriøs samtale om hans tilstand & Adam utløser pasientsignal & \begin{itemize}
\item Hva gjør du?
\item Hvor ser du?
\item Ønsker du å fortsette samtalen uavbrutt eller forlater du rommet? Hvilke faktorer spiller inn?
\item Hvordan påvirker signalet situasjonen?
\item Hvordan blir arbeidet påvirket?
\item Hvis du ikke velger å forlate rommet, hvorfor blir du?
\item Dersom du forlater rommet, og det viser seg at Adam ønsket et glass vann, føler du at du tok riktig avgjørelse?
\item Ville du brukt “kontakter” for å se de andre sin status
\item Om du merker at Jonas trenger at du er der en stund, ville du satt deg som opptatt? (rød)
\item Kommentarer til design?
\end{itemize}
& 14:25-14:40 & 15min\\
\end{tabular}\\
\caption{Scenario 3 for pasientsignal med prototype}
\label{Steg3.3}
\end{table}

\begin{table}[H]
\small
\begin{tabular}{p{3cm}|p{2cm}|p{4cm}|l|l}
\hline
\textbf{Scenario} & \textbf{Aksjon} & \textbf{Oppfølgingsspørsmål} & \textbf{Tidspunkt} & \textbf{Varighet}\\
\hline
Du er inne hos Adam og utfører sårskift. & Jonas utløser pasientsignal. \emph{Telefonen vibrerer} & \begin{itemize}
\item Hva gjør du?
\item Hvor ser du?
\item Ønsker du å fortsette sårskiftet eller forlater du rommet? Hvilke faktorer spiller inn?
\item Hvordan påvirker signalet situasjonen?
\item Hvordan blir arbeidet påvirket?
\item Hvis du ikke velger å forlate rommet, hvorfor blir du?
\item Ville du satt deg som utilgjengelig? Sagt fra til noen om at du skal foreta et sårskift?
\item Ville du brukt “kontakter” for å se de andre sin status
\item Kommentarer til design?
\end{itemize}
& 14:40-15:00 & 20min\\
\end{tabular}
\caption{Scenario 2 for pasientsignal med prototype}
\label{Steg3.4}
\end{table}

\textbf{Pause - 15min}

\pagebreak

\subsubsection{Steg 4: Fokusgruppe/Sammendrag - 15:15-16:00 (45min)}

\begin{itemize}
\item Oppsummering av scenarioene
	\begin{itemize}
	\item Var situasjonene som ble utspilt relevante? Eventuelt hvorfor ikke?
\item Forslag til oppfølgingsspørsmål:
	\item Tror dere noe av den løsningen vi foreslo kan være av nytte?
	\item Kan dere se for dere noe sånt i bruk?
	\item I forhold til å ta avgjørelsen om dere skal forlate pasienten til fordel for pasienten som ringer - synes dere prototypen ga tilstrekkelig informasjon?
		\item Er det annen informasjon dere kunne tenke dere?
	\item Tror dere informasjonen prototypen gir vil gjøre avbrytelsene mindre forstyrrende?
	\item Tror dere funksjonen om tilstedeværelsen til andre sykepleiere vil gjøre avbrytelsene mindre forstyrrende?
	\end{itemize}
\end{itemize}