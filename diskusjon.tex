\chapter{Diskusjon}
\label{chp:diskusjon}
I denne delen vil vi diskutere funnene fra workshopene som ble holdt i lys av teorien som er presentert i kapittel \ref{chp:teori}, og prøve å legge grunnlag for å besvare forskningsspørsmålene indrodusert i kapittel \ref{chp:introduksjon}.

\noindent
\section{Oversikt}
Det ble gjennom workshopene tydelig at sykepleierene i stor grad har god overikt over både pasienter og kolleger på tunet de jobber på. Randell (2010) trekker frem awareness i CSCW som "det å syneliggjøre sine aktiviteter for, og oppfatte aktivitetene til, kolleger, for å støtte opp om samarbeid dem imellom". I følge Heath et a. (2002) og Schmidt (2002) oppnås denne typen awareness gjennom kontinuerlig interaksjon med andre. Bardram og Hansen (2004) fortsetter denne tanken med å hevde at man med informasjonen man innhenter kan danne seg et bilde av hvordan stuasjonen til en hver tid er.

\noindent
\begin{itemize}
\item Alle typer redundans - CHECK (unntatt av innsats)
\item Kanskje kort nevne hvordan CSCW  handler om hvordan man jobber sammen og hvordan datasystemer kan støtte dette - bemanningsplan feks - felles mål tjo og hei.
\item Distribuert kognisjon - delt bevissthet av mål, planer og detaljer - dette kan støttes av kognitive artefakter som gjør informasjon offentlig tilgjengelig.
\item sosial translucence
\end{itemize}

\noindent
At alle sykepleierene som går på vakt skal være tilstede under vaktskiftemøtet fører til at alle får samme informasjon, og det oppstår en redundans av informasjon, eller data - som betyr at samme data er lagret på flere steder, her hos sykepleierene. Fordelingen av ansvar for pasienter mellom sykepleierene, hvor det er en primærsykepleier og minst en disponibel sykepleier per pasient, i kombinasjon med en grundig gjennomgang av pasientenes sykdomsbilde og tilstand, gir redundnas av funksjon (flere kan utføre samme oppgave), spesielt dersom vi antar at sykepleierene har stort sett de samme ferdightene, noe som er naturlig å anta da de har samme utdannelse. I følge Cabitza et al. (2005) kan awareness oppstå som et resultat av slik redundans, noe vi også ser gjennom måten sykepleierene kombinerer kunnskapen om pasientenes tilstand og hvem som har anvar for disse på for å danne seg et bilde av hvor kollegene sannsynligvis befinner seg og hva de gjør.
Gjennom vaktskiftemøtet får de også oversikt over hendelser som har skjedd i løpet av den siste tiden, hva som er status quo og over planlagte hendelser i fremtiden. Ut fra dette møtet har de derfor med seg en god kunnskapsbase å bygge videre på med informasjonen de får gjennom arbeidsdagen. 

\noindent
Ser vi på de tre typene awareness Bardram et al. (2006) mener er de viktigste i helseomsorgen ser vi at disse allerede er godt representert etter vaktskiftemøtet. Sosial awareness har de gjennom fordelingen av pasienter og psientenes sykdomsbilde gir en indikasjon på hvor sykeplierene sannsynligvis vil være mye av tiden, og hva slags aktiviteter de vil delta i. Romlig awareness får de også gjennom gjenomgangen av pasientenes sykdomsbilde og tilstand og da også hva som skal gjøres på de forskjellige pasientromene. Tidsmessig awareness tilegnes gjennom oversikten over tidligere og planlagte aktiviteter og status quo. Eksplisitt og implisitt infomasjon og oppdateringer sykepelierene får i løpet av dagen hjelper dem å opprettholde disse tre typene awareness. Det er en vanlig oppfattning blandt sykepleierene at dette er awareness alle har, noe som understrekes yttligere gjennom at de skjelden gir hverandre eksplisitt beskjed når man går til aktiviteter som kan ta lengre tid. 

\noindent
[DETTE AVSNITTET ER MULIGENS LITT STAKATO]
Randell (2010) trekker frem awareness i CSCW som "det å syneliggjøre sine aktiviteter for, og oppfatte aktivitetene til, kolleger for å støtte opp om samarbeid dem imellom". I følge Heath et al. (2002) og Schmidt (2002) oppnås denne typen awareness gjennom kontinuerlig interaksjon med andre. Bardram og Hansen (2004) fortsetter denne tanken med å hevde at man med informasjonen man innhenter kan danne seg et bilde av hvordan stuasjonen til en hver tid er.
Schmidt (2002) hevder også at kognitive artefakter er hensiktsmessig som støtte i innhenting og deling av slik informasjon i situasjoener hvor man ikke alltid er samlokalisert. Slike artefakter kan også være til nytte i forbindelse med distribuert kognisjon, definert av Nemeth er al. (2004) som den delte bevisstheten om mål planer og detaljer som ingen enkeltperson ikke kan begripe alene, jf \ref{DistrKogn}.
Oversiktsbildet over i hvilke rom det er sykepleier tilstede er et eksempel på en slik artefakt, og med bakgrunn i sykepleierenes streben etter at det er primærsykepleier som skal besvare en pasients utsløste signal kan vi gå ut ifra at sykepleierene antar at det er primærsykepleier, eventuelt disp, som er tilstede når det vises tilstedeværelse i dette skjermbildet, dersom de ikke innehar informasjon som taler imot dette.




\noindent
Etter introduksjon av prototypen, og muligheten til å sette sin status til utilgjengelig
for å formidle nettopp dette til kolleger så vi at dette var en funksjon som ble brukt
i større grad enn å gi muntlig beskjed. Hva som var grunnen til dette er uklart, men 

\noindent
At det ble stilt spørsmålstegn ved hvordan prototypen eventualt ville sikre at sykepleiere ikke ble vist som utilgjengelig dersom de ikke var det viser oss at de er klar over hvor viktig denne awarenessen er i deres daglige arbeid. 


\section{Avbrudd og kognitiv kapasitet}


