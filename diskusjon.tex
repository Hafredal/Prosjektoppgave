\chapter{Diskusjon}
\label{chp:diskusjon}
I denne delen vil vi diskutere funnene fra workshopene som ble holdt i lys av teorien som er presentert i kapittel \ref{chp:teori}, og prøve å legge grunnlag for å besvare forskningsspørsmålene indrodusert i kapittel \ref{chp:introduksjon}.

\noindent
\section{Oversikt}
Sykepleierenes generelle oversikt over inneliggende pasienter og over hvilke sykepleiere som har primæransvar for disse, og deres evne til å bruke denne kunnskapen til å i stor grad kunne lokalisere kolleger uten større vanskeligheter vitner om god awareness. 

\noindent
Gjennom vaktskiftemøtet får de oversikt over hendelser som har skjedd i løpet av den siste tiden, hva som er status quo og over planlagte hendelser i fremtiden. Også pasientene og deres tilstand blir grundig gjennomgått under møtet i tillegg til at det blir fordelt primæransvar og diponible sykepleiere til hver pasient. Fordelingen av ansvar gir også en forståelse for hvilke aktiviteter de forskjellige sykepleierene vil bli engasjert i utifra kunnskapen om pasientene. Ut fra dette møtet har de derfor med seg en god kunnskapsbase å bygge videre på med informasjonen de får gjennom arbeidsdagen. 

\noindent
Ser vi på de tre typene awareness Bardram et al. (2006) mener er de viktigste i helseomsorgen ser vi at disse allerede er godt representert etter vaktskiftemøtet. Sosial awareness har de gjennom fordelingen av pasienter og psientenes sykdomsbilde gir en indikasjon på hvor sykeplierene sannsynligvis vil være mye av tiden, og hva slags aktiviteter de vil delta i. Romlig awareness får de også gjennom gjenomgangen av pasientenes sykdomsbilde og tilstand og da også hva som skal gjøres på de forskjellige pasientromene. Tidsmessig awareness tilegnes gjennom oversikten over tidligere og planlagte aktiviteter og status quo. Eksplisitt og implisitt infomasjon og oppdateringer sykepelierene får underveis dagen hjelper dem å opprettholde disse tre typene awareness. Det er en vanlig oppfattning blandt sykepleierene at dette er awareness alle har, noe som understrekes yttligere gjennom at de skjelden gir hverandre eksplisitt beskjed når man går til aktiviteter som kan ta lengre tid. 

\noindent
I dag kommer også mye av den sosiale awarenessen fra bemanningsplanen, og oversiktsbildet over hvilke rom hvor det er sykepleier tilstede. Dette oversiktsbildet er en form for kognitiv artifakt som 

Med bakgrunn i sykepleierenes streben etter at det er primærsykepleier som skal besvare en pasients utsløste signal kan vi gå ut ifra at sykepleierene antar at det er primærsykepleier, eventuelt disp, som er tilstede når det vises tilstedeværelse i dette skjermbildet, dersom de ikke innehar informasjon som taler imot dette. 

\noindent
At det ble stilt spørsmålstegn ved hvordan prototypen eventualt ville sikre at sykepleiere ikke ble vist som utilgjengelig dersom de ikke var det viser oss at de er kalr over hvor viktig denne awarenessen er i deres daglige arbeid. 


\section{Avbrudd og kognitiv kapasitet}
\noindent
Selv med relativt god awareness på flere områder kommer det frem at de ikke har oversikt over hvor alle er til enhver tid, noe som er logisk da sykepleierene ikke oppholder seg på sengetunet til en hver tid, og at de ofte bruker telefon for å få tak i hverandre dersom de ikke er på de forventede plassene. Bruken av telefon på denne måten er en av mange grunner til de mange avbruddene en sykepleier opplever i løpet av en dag 
