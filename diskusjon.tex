\chapter{Diskusjon}
\label{chp:diskusjon}
I denne delen vil vi diskutere funnene fra workshopene som ble holdt, i lys av teorien som er presentert i kapittel \ref{chp:teori}, og legge grunnlaget for å besvare forskningsspørsmålene indrodusert i kapittel \ref{chp:introduksjon}.

\section{Valg av teori}
Som nevnt i kapittel \ref{subsec:tidligereArbied} fikk vi innledningsvis utlvert flere artikler av vår veileder. Det er ikke til å komme forbi at disse satte utgangspunktet for veien videre. I hvor stor grad oppgaven ville tatt en annen retning dersom vi hadde tatt utgangspunkt i andre artikler er uvisst, men da vi allerede hadde en oppgavebeskrivelse å gå ut ifra ser vi det som usannsynlig at forskjellene ville vært store.
Der vi fant temaer vi ønsket å tilegne oss mer kunnskap om ble videre teori først valgt på bakgrunn av kildene til artiklene vi allerede hadde lest. Dersom det var behov for videre utfyllende teori tok vi i bruk Googles søkemotor Google Scholar, en søkemotor for akademisk litteratur. 

\noindent
Det har vært viktig for oss at kildene vi har valgt er pålitelige, og vi har dermed fosrsøkt å holde oss til artikler som er blitt publisert i tidsskrifter. Der det er brukt elektroniske kilder (nettsider) er dette dokumenter for spesifikke fakta om blandt annet St. Olavs Hospital, samt brukerveiledninger for pasientsignalsystemt som brukes i dag. 

\section{Prototypen}
\label{protoDisk}
Funksjonaliteten til prototypen ble i hovedsak basert på teori fra artiklene vi har lest, og hva vi utifra dette anså som mulige forbedringer. Vi fikk noe veiledning fra professor og veileder, men hadde ingen direkte innspill fra brukerene av systemet som grunnlag for valgene vi tok underveis.
Vi ønsket likevel å lage en prototyp til et system som ville blitt brukbart ved implementering (jf \ref{chp: brukbarhet}). Siden det var svært vanskelig å si hva brukeren forventet og hva som måtte til for å være effektiv i bruk, ble det mye synsing og bruk av det vi mente var sunn fornuft. Vi fokuserte på at all informasjon skulle være få trykk unna og at skjermbildene skulle ha samme oppbygning. Videre sørget vi for at viktig informasjon, som brukerens egen status, var direkte aksesserbart fra alle skjermbildene. 

\noindent
I dag brukes de trådløse enhetene i relativt liten grad for å støtte opp om awareness, da det er stort sett oversiktsbildet over på hvilke rom det er sykepleiere tilstede som brukes til dette. Med prototypen vi har laget er det lagt opp til langt mer omfattende bruk av de personlige enhetene til dette formålet. Det påpekes av Erickson og Kellog (2000) at sosialt translucent er en svært viktig egenskap ved et slikt system (jf \ref{awareness_CSCW}), noe vi vil argumentere for at vår prototype oppfyller.  Dersom en sykepleier setter sin status til utilgjengelig vil dette straks distribueres til kollegenes telefoner. På denne måten syneliggjøres det at de er engasjert i en aktivitet de ikke helst ikke vil forstyrres i. Dette gjør det mulig for andre sykepleiere å ta ansvarlige valg angående egne aktiviteter, samt i forhold til om se skal avbryte sykepleieren som er utilgjengelig. 

\noindent
Da det ikke var vår intensjon å teste skjermbildedesign til en slik enhet, la vi heller ikke vekt på bruk av teorier for dette. Likevel brukte vi tid på å lage et design vi trodde ville falle i smak hos deltagerene av workshopene. Dette var først og fremst for at dårlig design ikke skulle bli en faktor som påvirket tesingen av prototypens funktionalitet i stor grad. Om dette hadde noe for seg er det vanskelig å si noe om, men da dette ikke ble kommentert velger vi å anta at designet ikke var i veien for tesingen.

\section{Funn}

\subsection{Oversikt}
\label{oversikt}
Workshopene avdekket at sykepleierestudentene opplever at de har god oversikt over både pasienter og kolleger på tunet de jobber på. Ved vaktskiftet fordeles ansvar for pasientene, og informasjon om deres sykdomsbilde og tilstand blir gitt. Vi kan dermed argumentere for at vaktskiftemøtet resulterer i en form for kognitiv distribusjon, jf \ref{DistrKogn}, da sykepleierene som gruppe har større kapasitet til å holde en detaljert oversikt over pasientene, enn en sykepleier kan alene. Med kunnskapen sykepleierene tilegner seg ved dette møtet, oppstår redundans av funksjon, da flere sykepleiere kan hjelpe de samme pasientene. Vaktskiftemøtet resulterer også i sosial awareness, da sykepleierene får en indikasjon på hvilke aktiviteter deres kolleger vil delta i, og dermed hvor de sannsynligvis vil være å finne. Likevel poengterer S1-A1 at sykepleierene ikke vet hvor alle er til enhver tid, da det er stor variasjon i hvor vanlig det er å varsle andre om sine aktiviteter. Dette kan tyde på at sykepleierene ikke anser det som nødvendig å si i fra, da de antar at deres kolleger vet hva de gjør og hvor de er, selv om dette nødvendigvis ikke er tilfellet. Awareness slik det defineres innen CSCW, er å synliggjøre sine aktiviteter for, og oppfatte aktivitetene til kolleger, for å støtte opp om samarbeidet dem imellom. Ifølge Heath et al. (2002) og Schmidt (2002) oppnås denne typen awareness gjennom kontinuerlig interaksjon med andre. Vi ser derfor at vaktskiftemøtet fører til en viss sosial awareness, men at denne ikke alltid opprettholdes da det er variasjon i hvor stor grad man eksplisitt synliggjør sine aktiviteter. Ved gjennomgangen av pasientenes sykdomsbilde og tilstand, får sykepleierene et inntrykk av  hva som skal gjøres, eller hva som skjer på de forskjellige pasientrommene. Dette kan dermed tyde på at sykepleierene har god romlig awareness, såfremt det ikke oppstår situasjoner hvor pasienter bytter rom, eller uforutsette hendelser oppstår. Et eksempel på dette gis av S5-A3 som forklarer hvorfor man ikke sier i fra om at man skal inn på et rom hvor man kan bli værende en stund, med at de andre allerede er klare over hva som skjer. På samme måte tilegnes tidsmessig awareness gjennom denne oversikten som blir gitt over tidligere og planlagte aktiviteter, samt status quo. Vi ser dermed at de tre typene awareness Bardram et al. (2006) mener er de viktigste i helseomsorgen, til en viss grad er tilstede.

\noindent
Kognitive gjenstander, som tavler, lister og skjermer, beskrives som hensiktsmessige for å støtte deling og innhenting av informasjon. Sykepleierenes skriftlige oversikt over pasienter kan derfor sies å være en slik gjenstand, disse er derimot private og gir kun informasjon til de enkelte sykepleierene. Som et resultat av dette, oppstår redundans av data, da informasjon om pasientene finnes både i deres journaler, og notert i sykepleierenes lister. 

\subsection{Avbrudd og kognitiv kapasitet}
Da avbrudd kan være både positive og negative blir det viktig å gjøre en avveining i forhold til hvorvidt sykepleierene bør varsles, og eventuelt på hvilken måte dette bør gjøres. Under workshopene så vi at det var stor variasjon i hvordan sykepleierene ble påvirket av innkommende pasientsignaler. For hvert signal, måtte sykepleierene gjøre en vurdering på om de skulle bli, eller forlate pasienten de var inne hos. I tråd med Ebright (2010), som hevder at denne re-prioriteringen kan resultere i en kognitiv belastning som kan hemme oppmerksomheten, er det grunn til å anta at alle sykepleierene i større eller mindre grad blir påvirket av signalene. Det var derimot stor variasjon i hvorvidt sykepleierene selv opplevde å bli påvirket, da noen uttrykte at ikke lot seg påvirke, mens andre sa de ble stresset.

\noindent
Hvordan vasler og lyder kunne endres, og spesielt dersom sykepleieren valgte å sette sin status til utilgjengelig, ble derfor naturlig å diskutere under worskshopene. Grandhi og Jones (2010) beskriver fire teknikker for å håndtere avbrudd med hansikt å redusere negative effekter samtidig fom de positive effektene beholdes. Forslaget om å hoppe over sykepleiere med status utilgjengelig i første omgang kan knyttes opp mot teknikken forebygging, i form av at avbrytelsen da blir blokkert. Siden det ikke gis eksplisitt informasjon gjennom systemet om hva slags oppgave sykepleieren utfører, heller ikke informasjon om hva avbruddet gjelder, vil det ikke være aktuelt å kun tillate visse anrom basert på relevans i forhold til oppgave. Videre kan man si at muligheten til å sette status til utilgjengelig også vil ta i bruk teknikken fraråding, da symbolet som viser at en sykepleier er utilgjengelig og ønsker å ikke bli forstyrret implisitt fraråder andre sykepleiere å avbryte vedkomne. Her må hver enkelt selvfølgelig ta en vurdering på om innholdet i avbrytelsen er så viktig at det overgår den negative effekten ved avruddet. Endring i ringevolum fanges opp av teknikken modifisering, og reduserer interferensen mellom perseptuelle og kognitive prosesser til et minimum.
At sykeplierene vet hvilket rom som har utløst signalet, sammen med deres kunnskap om tilstanden til pasienten er viktig for hvordan de prioriterer signalet. Denne informasjonen fungerer da som en forhondsvisning. Selv om det i dette tilfellet ikke gir direkte informasjon angående hva signalet gjelder, vil romnummeret sammen med sykepleierens kunnskap om blandt annet pasientens tilastnad gi en indirekte indikasjon. Muligheten for å gå pasientene flere signalknapper for bruk ved forskjellige behov (som for eksempel drikke, toalettbesøk eller mer akutt hjelp), ble diskutert under workshopene, men deltagerene kom frem til at i tillegg til mulighet for misbruk fra pasientenes side, og faren for at sykepleierene skulle nedpriorietere signaler om mindre akutte behov, som drikke, er faren for at dersom en pasient skulle få et illebefinnende, og den første signalknappen denen fikk tak på var vannglasset kunne dette få fatale følger. Et annet forslag som kom frem på den andre dagen var et oversiktsbilde over rommet hvor signalet var utløst. Det var bred enighet om at dette ville være et godt grunnlag for å vurdere prioriteringen av signalet.   

\noindent
Etter introduksjon av prototypen, og muligheten til å sette sin status til utilgjengelig
for å formidle nettopp dette til kolleger så vi at dette var en funksjon som ble brukt
i større grad enn å gi muntlig beskjed. Det kan være flere grunner til dette, blandt annet at gjennom telefonen blir denne informasjonen distribuert til alle aktuelle sykepleiere, i stedet for bare dem som tilfeldigvis er tilstede der og da. Dette er også en mindre avbrytende måte å dele informasjon på, og er dermed med på å redusere mengden avbrytelser, som igjen kan hindre overbelastning av den enkeltes kognitive kapasitet. 

\noindent
Relasjonenen mellom sykepleieren som blir avbrutt av pasientsignalet og den pasienten som utløste signalet vil ha mye å si for hvordan sykepleieren prioriterer signalet. Dette er i tråd med Harr og Kaptelinins (2007) tanker om at oppførselen til avbryter og den avbrutte avhenger av mellommenskelige relasjoner. Selv om vi kan anta at pasientens adferd, eller valg om å avbrye eller ikke, ikke vil bli påvirket i nevneverdig grad i denne situasjonen, kom det tydelig frem under workshopene at dette for sykepleierene ofte er avgjørende for valget de tar i forhold til å besvare signalet eller ikke. 
Deres bekymring for hvordan den enkelte pasientent blir påvirket av at sykepleieren får et innkomne pasientsignal var tydelig. At også pasienten blir forstyrret når sykeplieren som er hos denne blir avbrutt, kalles av Harr og Kaptelinin (2007) for loksajonsbasert forstyrrelse. 

