\chapter{Diskusjon}
\label{chp:diskusjon}
I denne delen vil vi diskutere funnene fra workshopene som ble holdt, i lys av teorien som er presentert i kapittel \ref{chp:teori}, slik at vi kan besvare forskningsspørsmålene indrodusert i kapittel \ref{chp:introduksjon}.

\section{Valg av teori}
Som nevnt i kapittel \ref{subsec:tidligereArbied} fikk vi innledningsvis utlevert flere artikler av vår veileder, og det er ikke til å komme utenom at disse ble utgangspunktet for veien videre. I hvor stor grad oppgaven ville tatt en annen retning dersom vi hadde tatt utgangspunkt i andre artikler er uvisst, men da vi allerede hadde en oppgavebeskrivelse å gå ut ifra ser vi det som usannsynlig at forskjellene ville vært store.
Der det var temaer vi ønsket å tilegne oss mer kunnskap om ble videre teori valgt på bakgrunn av kildene til artiklene vi hadde lest. Dersom det var behov for utfyllende teori brukte vi Googles søkemotor for akademisk litteratur, Google Scholar. 

\noindent
Vi har i stor grad forsøkt å holde oss til artikler publisert i tidsskrifter da vi anser disse for å være pålitelige. Der det er brukt elektroniske kilder (nettsider) er dette dokumenter for spesifikke fakta, samt brukerveiledninger for pasientsignalsystemet som brukes ved St. Olavs Hospital. 

\section{Prototypen}
\label{protoDisk}
Funksjonaliteten til prototypen er i hovedsak basert på teori fra artiklene vi har lest, og hva vi utifra disse tolket som mulige forbedringer. Vi fikk veiledning av professor og veileder, men hadde ingen direkte innspill fra brukerne av systemet som grunnlag for valgene vi tok underveis.
Vi ønsket likevel å lage en prototype av et system som ville være brukbart ved implementering (jf \ref{chp: brukbarhet}). Siden det er svært vanskelig å vite hva brukerne forventer og hvilke oppgaver systemet skal være til støtte for, ble det gjort flere antagelser. Vi fokuserte på at ønsket informasjon skulle være få trykk unna og at skjermbildene skulle ha samme oppbygning.

\noindent
I prototypen er det lagt opp til mer omfattende bruk av de personlige enhetene til å synliggjøre og distribuere informasjon som kan være til støtte for awareness. Det påpekes av Erickson og Kellog (2000) at sosial translucence er viktig i slike systemer (jf \ref{awareness_CSCW}). Vi vil argumentere for at vår prototype innehar disse egenskapene.  Dersom en sykepleier setter sin status til utilgjengelig vil dette straks distribueres til kollegenes telefoner. På denne måten synliggjøres det at de er engasjert i en aktivitet de helst ikke vil forstyrres i. Dette gjør det mulig for andre sykepleiere å ta ansvarlige valg angående egne aktiviteter, spesielt i forhold til om en skal avbryte sykepleieren som er utilgjengelig. 


\section{Funn fra workshops}

\subsection{Oversikt}
\label{oversikt}
Workshopene avdekket at sykepleierstudentene opplever å ha god oversikt over både pasienter og kolleger på tunet de jobber på. Ved vaktskifte fordeles ansvar for pasientene, og informasjon om deres sykdomsbilde og tilstand blir gitt. Vi kan dermed argumentere for at vaktskiftemøtet resulterer i en form for kognitiv distribusjon, jf \ref{DistrKogn}, da sykepleierne som gruppe har større kapasitet til å holde en detaljert oversikt over pasientene, enn en sykepleier har alene. Med kunnskapen sykepleierne tilegner seg gjennom dette møtet, oppstår redundans av funksjon, da flere sykepleiere kan hjelpe de samme pasientene. Vaktskiftemøtet resulterer også i sosial awareness, da sykepleierne får en indikasjon på hvilke aktiviteter deres kolleger vil delta i, og dermed hvor de sannsynligvis vil være å finne. Likevel poengterte S1-A1 at sykepleierne ikke vet hvor alle er til enhver tid, og det er stor variasjon i hvor vanlig det er å varsle andre om sine aktiviteter. Dette kan tyde på at sykepleierne ikke anser det som nødvendig å si ifra, da de antar at deres kolleger vet hva de gjør og hvor de er, selv om dette ikke nødvendigvis er tilfellet. Awareness slik det defineres innen CSCW, er å synliggjøre sine aktiviteter for, og oppfatte aktivitetene til kolleger, for å støtte opp om samarbeidet dem imellom. Ifølge Heath et al. (2002) og Schmidt (2002) oppnås denne typen awareness gjennom kontinuerlig interaksjon med andre. Vi ser derfor at vaktskiftemøtet fører til en viss sosial awareness, men at denne ikke alltid opprettholdes da det er variasjon i hvor stor grad man eksplisitt synliggjør sine aktiviteter. Ved gjennomgangen av pasientenes sykdomsbilde og tilstand, får sykepleierne et inntrykk av  hva som skal gjøres, eller hva som skjer på de forskjellige pasientrommene. Dette kan dermed tyde på at sykepleierne har god romlig awareness, såfremt det ikke oppstår situasjoner hvor pasienter bytter rom, eller uforutsette hendelser oppstår. Et eksempel på dette ble gitt av S5-A3 som forklarte hvorfor man ikke sier ifra om at man skal inn på et rom hvor man kan bli værende en stund, med at de andre allerede er klar over hva som skjer. På samme måte tilegnes tidsmessig awareness gjennom denne oversikten som blir gitt over tidligere og planlagte aktiviteter. Vi ser dermed at de tre typene awareness Bardram et al. (2006) mener er de viktigste i helseomsorgen, er tilstede til en viss grad.

\noindent
Kognitive gjenstander, som tavler, lister og skjermer, beskrives som hensiktsmessige for å støtte deling og innhenting av informasjon. Sykepleiernes skriftlige oversikt over pasienter kan derfor sies å være en slik gjenstand, disse er derimot private og gir kun informasjon til de enkelte sykepleierne. Som et resultat av dette, oppstår redundans av data, da informasjon om pasientene finnes både i deres journaler, og notert i sykepleiernes lister. Som S4-A3 foreslo, kan mer informasjon om pasienten som utløser signalet til en viss grad erstatte denne listen.

\noindent
Deltagerne under WS1 var i stor grad enige om at den automatiske tilstedemarkeringen vil være en forbedring i forhold til situasjonen i dag, hvor flere glemmer å trykke seg inn. Denne automatiseringen er en form for biprodukt-awareness, (jf. \ref{chp: awareness}), da aktiviteten synliggjøres uten at det krever merarbeid for sykepleieren, i motsetning til slik det er i dag, hvor sykepleieren selv må markere seg som tilstede. En slik automatikk vil føre til at denne informasjonen i større grad er pålitelig, og som påpekt av S3-A3 kan denne informasjonen være til støtte ved håndtering av innkommende pasientsignal. Vi observerte derimot ikke at noen av deltagerne brukte denne informasjonen da scenarioene med prototype ble utspilt. Om dette var fordi de likevel ikke anså denne informasjonen som nyttig, eller om det var andre forhold som påvirket deltagernes handlinger er vanskelig å avgjøre. Vi så i ettertid av workshopene at scenarioene kan ha manglet tilstrekkelig realisme i form av arbeidsflyt, og at vi hadde noe høye forventninger til at deltagerne skulle føle seg komfortable med bruk av prototypen i løpet av den korte tiden. 

\subsection{Avbrudd og kognitiv kapasitet}
Da avbrudd kan ha både positive og negative effekter er det viktig å gjøre en avveining i forhold til hvorvidt sykepleierne bør varsles, og eventuelt på hvilken måte dette bør gjøres. Under workshopene så vi at det var stor variasjon i hvordan sykepleierne ble påvirket av innkommende pasientsignaler. For hvert signal, måtte sykepleierne gjøre en vurdering på om de skulle bli, eller forlate pasienten de var inne hos. I tråd med Ebright (2010), som hevder at denne re-prioriteringen kan resultere i en kognitiv belastning som kan hemme oppmerksomheten, er det grunn til å anta at alle sykepleierne i større eller mindre grad blir påvirket av signalene. Det var derimot stor variasjon i hvorvidt sykepleierne selv opplevde å bli påvirket.

\noindent
Relasjonen mellom sykepleieren som mottar pasientsignalet og den pasienten som utløste signalet vil ha mye å si for hvordan sykepleieren prioriterer signalet. Dette er i tråd med Harr og Kaptelinins (2007) tanker om at oppførselen til avbryter og den avbrutte avhenger av mellommenskelige relasjoner. Selv om vi kan anta at pasientens adferd, eller valg om å avbryte eller ikke, ikke vil bli påvirket i nevneverdig grad i denne situasjonen, kom det tydelig frem under workshopene at dette for sykepleierne ofte er avgjørende for valget de tar i forhold til å besvare signalet eller ikke. At sykepleierne vet hvilket rom som har utløst signalet, sammen med deres kunnskap om tilstanden til pasienten er viktig for hvordan de prioriterer signalet. Denne informasjonen fungerer da som en forhåndsvisning. Selv om det i dette tilfellet ikke gir direkte informasjon angående hva signalet gjelder, vil romnummeret sammen med sykepleierens kunnskap om blant annet pasientens tilastnad gi en indirekte indikasjon. 
 Muligheten for å gi pasientene flere signalknapper for bruk ved forskjellige behov (som for eksempel drikke, toalettbesøk eller mer akutt hjelp), ble diskutert under workshopene, men deltagerne kom frem til at i tillegg til mulighet for misbruk fra pasientenes side, og faren for at sykepleierne skulle nedpriorietere signaler om mindre akutte behov, som drikke, er faren for at dersom en pasient skulle få et illebefinnende, og den første signalknappen denen fikk tak på var vannglasset kunne dette få fatale følger. Et annet forslag som kom frem på den andre dagen var et oversiktsbilde over rommet hvor signalet var utløst. Det var bred enighet om at dette ville være et godt grunnlag for å vurdere prioriteringen av signalet. 
 
\noindent
Etter at prototypen og dens funksjon for å sette status til utilgjengelig var introdusert, så vi at dette var en funksjon som ble brukt i større grad enn å gi muntlig beskjed. Dette kan gi flere positive effekter, blant annet at informasjonen blir distribuert til alle sykepleierne på tunet, på en mindre avbrytende måte. Dette vil kunne redusere mengden avbrytelser, da sykepleierne i som igjen kan hindre overbelastning av den enkeltes kognitive kapasitet. Det var derimot ingen av deltagerne som ville være fullstendig isolert fra innkommende pasientsignaler, da de ønsker en viss oversikt over hva som skjer. Som påpekt av Harr og Kaptelinin (2007) er nettopp denne avveiningen mellom det å være isolert, og dermed miste informasjon, eller å bli avbrutt, en av de fundamentale utfordringene ved avbruddshåndtering. Videre kan man si at muligheten til å sette status til $"$utilgjengelig$"$ også vil ta i bruk teknikken fraråding, da symbolet som viser at en sykepleier er utilgjengelig og ønsker å ikke bli forstyrret implisitt fraråder andre sykepleiere å avbryte vedkomne. Her må hver enkelt selvfølgelig ta en vurdering på om innholdet i avbrytelsen er så viktig at det overgår den negative effekten ved avruddet. 

\noindent
Hvordan varsler og lyder kunne endres, og spesielt dersom sykepleieren valgte å sette sin status til utilgjengelig, ble derfor naturlig å diskutere under worskshopene.
Ved å endre eller modifisere hvordan pasientsignaler varsles vil interferensen mellom perseptuelle og kognitive prosesser kunne reduseres, og dermed påvirke oppgaveytelsen i mindre grad. Dette vil også være positivt for pasientene, da sykepleierne uttrykte bekymring for hvordan den enkelte pasientent blir påvirket av varslingen, som er en form for lokasjonsbasert forstyrrelse (jf. \ref{chp:avbrudd}). 

\noindent
Forslaget om å hoppe over sykepleiere med status utilgjengelig i første omgang kan sies å være en form for forebygging, da avbrytelsen blir blokkert. Siden det ikke gis eksplisitt informasjon gjennom systemet om hva slags oppgave sykepleieren utfører, heller ikke informasjon om hva avbruddet gjelder, vil det ikke være aktuelt å kun tillate visse signaler basert på relevans til oppgave. 

