\chapter{Diskusjon}
\label{chp:diskusjon}
I denne delen vil vi diskutere funnene fra workshopene som ble holdt, i lys av teorien som er presentert i kapittel \ref{chp:teori}, og legge grunnlaget for å besvare forskningsspørsmålene indrodusert i kapittel \ref{chp:introduksjon}.

\section{Prototypen}
\label{protoDisk}
Som nevnt i kapittel \ref{subsec:tidligereArbied} fikk vi utlvert flere artikler av vår veileder i starten av prosessen. Det er ikke til å komme forbi at artiklene vi startet med på denne måten satte utgangspunktet for veien videre. I hvor stor grad oppgaven ville tatt en annen retning dersom vi hadde startet med andre artikler er uvisst, men da vi allerede hadde en oppgavebeskrivelse å gå ut ifra ser vi det som usannsynlig at forskjellene ville vært store.
Der vi fant temaer vi øsnket å tilegne oss mer kunnskap om ble videre teori valgt på bakgrunn av kildene til artiklene vi allerede hadde lest. 

\noindent
Prototypen vi laget ble basert på teori fra artiklene vi leste. Vi fikk noe veiledning fra professor og veileder, men hadde ingen direkte innspill fra brukere av systemet å basere valgene vi tok på. 

PROTOTYPE BASERT PÅ TEORI

\noindent
I dag brukes CSCW for støtte om awareness i relativt liten grad, da det er stort sett oversiktsbildet over hvor det er sykepleiere tilstede som brukes til dette. Med prototypen vi har laget er det lagt opp til langt mer bruk av denne til å støtte opp om sykepleiernes awareness. At et slikt system er sosialt translucent er i følge Erickson og Kellog (2000) svært viktig (jf \ref{awareness_CSCW}), et krav vi vil argumentere for at vår prototype oppfylle. Dersom en sykepleier setter sin status til utilgjengelig vil dette straks distribueres til deres kollegers telefoner, og på denne måten syneliggjøre, om ikke hvilken aktivitet de gjør, men at de er engasjert i en aktivitet de ikke helst ikek vil forstyrres i. Dette gjør det mulig for andre sykepleiere å ta ansvarlige valg angående egne aktiviteter, samt i forhold til om å avbryte sykepleieren som er utilgjengelig. Oversikten over hvem som er tilgjengelige og utilgjengelige støtter også opp om sykepleierenes sosiale awareness.

\section{Funn}

\subsection{Oversikt}
\label{oversikt}
Det ble gjennom workshopene tydelig at sykepleierene i stor grad har god overikt over både pasienter og kolleger på tunet de jobber på. Randell (2010) trekker frem awareness i CSCW som "det å syneliggjøre sine aktiviteter for, og oppfatte aktivitetene til, kolleger, for å støtte opp om samarbeid dem imellom". I følge Heath et al. (2002) og Schmidt (2002) oppnås denne typen awareness gjennom kontinuerlig interaksjon med andre. Bardram og Hansen (2004) fortsetter denne tanken med å hevde at man med informasjonen man innhenter kan danne seg et bilde av hvordan stuasjonen til en hver tid er.

\noindent
At alle sykepleierene som går på vakt skal være tilstede under vaktskiftemøtet fører til at alle får samme informasjon, og det oppstår en redundans av informasjon, eller data - som betyr at samme data er lagret på flere steder, her hos sykepleierene. Fordelingen av ansvar for pasienter mellom sykepleierene, hvor det er en primærsykepleier og minst en disponibel sykepleier per pasient, i kombinasjon med en grundig gjennomgang av pasientenes sykdomsbilde og tilstand, gir redundnas av funksjon (at flere kan utføre samme oppgave), spesielt dersom vi antar at sykepleierene har stort sett de samme ferdightene, noe som vil være naturlig da de har samme utdannelse. I følge Cabitza et al. (2005) kan awareness oppstå som et resultat av slik redundans, noe vi også ser gjennom måten sykepleierene kombinerer kunnskapen om pasientenes tilstand og hvem som har anvar for disse, for å danne seg et bilde av hvor kollegene sannsynligvis befinner seg, og hva de gjør.
Gjennom vaktskiftemøtet får de også oversikt over hendelser som har skjedd i løpet av den siste tiden, hva som er status quo og over planlagte hendelser i fremtiden. Ut fra dette møtet har de derfor med seg en god kunnskapsbase å bygge videre på med informasjonen de får gjennom arbeidsdagen. 

\noindent
Ser vi på de tre typene awareness Bardram et al. (2006) mener er de viktigste i helseomsorgen ser vi at disse allerede er godt representert etter vaktskiftemøtet. Sosial awareness har sykepleierene gjennom fordelingen av pasienter og pasientenes sykdomsbilde, noe som gir en indikasjon på hvor sykeplierene sannsynligvis vil være mye av tiden, og hva slags aktiviteter de vil delta i. Romlig awareness får de også gjennom gjenomgangen av pasientenes sykdomsbilde og tilstand og da også hva som skal gjøres på de forskjellige pasientrommene. Tidsmessig awareness tilegnes gjennom oversikten over tidligere og planlagte aktiviteter og status quo. Eksplisitt og implisitt infomasjon og oppdateringer sykepelierene får i løpet av dagen hjelper dem å opprettholde disse tre typene awareness. Det er en vanlig oppfattning blandt sykepleierene at dette er awareness alle har, noe som understrekes yttligere gjennom at de skjelden gir hverandre eksplisitt beskjed når de går til aktiviteter som kan ta lengre tid. 

\noindent
Selv om alle sykepleierene får dan samme informasjonen under vaktskiftemøtet, og i så måte har samme kunnskap, vil det faktum at en sykepleier gjerne har ansvar for de samme pasiente gjennom flere vakter vil ofte sykepleierene ha mer detaljert kunnskap om sine pasienter enn om andres. Vi kan da argumentere for at dette er en form for kognitiv distrubisjon, definert av Nemeth et al. (2004) som den delte bevisstheten om mål planer og detaljer som ingen enkeltperson ikke kan begripe alene, jf \ref{DistrKogn}. Dette fordi sykepleierene da som gruppe har en svært detaljert oversikt over pasientene, i motsettning til enkeltvis hvor de ikke har kapasitet til å kjenne alle pasientene på denne måten. 
Schmidt (2002) hevder at kognitive artefakter vil være hensiktsmessig som støtte i innhenting og deling av informasjon i situasjoener hvor man ikke alltid er samlokalisert. Slike artefakter kan også være til nytte i forbindelse med distribuert kognisjon, 
Slike artefakter vil da fungere som en støtte for awareness slik den er beskrevet av Randell (2010) (se begynnelsen av kapittel \ref{oversikt}) og for den distribuerte awarenessen beskrevet av Nemeth et al. (2004).
Eksempler på slike kognitive artefakter er kan være oversiktsbildet som viser i hvilke rom det er sykepleier tilstede, og med bakgrunn i sykepleierenes streben etter at det er primærsykepleier som skal besvare en pasients utsløste signal kan vi gå ut ifra at sykepleierene antar at det er primærsykepleier, eventuelt disp, som er tilstede når det vises tilstedeværelse i dette skjermbildet, dersom de ikke innehar informasjon som taler imot dette. Et annet eksempel er den skriftlige oversikten over inneliggende pasienter sykepleierene får under vaktskiftemøtet, og som de til en hver tid har med seg.

\noindent
At det ble stilt spørsmålstegn ved hvordan prototypen eventuelt ville sikre at sykepleiere ikke ble vist som utilgjengelig dersom de ikke var det, og dermed skape barrierer i stedet for translucence, viser oss at de er klar over hvor viktig denne awarenessen er i deres daglige arbeid. 


\subsection{Avbrudd og kognitiv kapasitet}
Avbrudd og hvordan disse påvirker den enkeltes sykepleiers kognitive kapasitet er vesentlig å ta hensyn til da overbelstning av denne kan gjøre sykepleierene uoppmerksome. En nøye beskrivelse av denne kapasiteten er gitt i kapittel \ref{chp: kognisjon}, mens avbrudd, og deres dualitet er beskrvet i kapittel \ref{chp: avbrudd}. 

\noindent
Da avbrudd kan være både positive og negative blir det viktig å gjøre en avveining i forhold til hvorvidt det skal avbrytes eller ikke og på hvilken måte dette i så fall kan gjøres på en mest mulig skånsom måte. Under workshopene så vi at det var stor variasjon i hvordan sykepleierene ble påvirket av innkomne pasientsignal. Hos de deltagerene som ble stresset og begynte å tenke på det nye pasientsignalet, hva det gjaldt og hvordan det påvirket pasienten de var hos, kan vi anta at arbeidsminnet i stor grad vil bli overbelastet, som igjen kan føre til at sykepleieren blir ukonsenrert og for eksempel glemmer oppgaver som skal gjøres. 

\noindent
Hvordan vasler og lyder kunne endres, og spesielt dersom sykepleieren valgte å sette sin status til utilgjengelig, ble derfor naturlig å diskutere under worskshopene. Grandhi og Jones (2010) beskriver fire teknikker for å håndtere avbrudd med hansikt å redusere negative effekter samtidig fom de positive effektene beholdes. Forslaget om å hoppe over sykepleiere med status utilgjengelig i første omgang kan knyttes opp mot teknikken forebygging, i form av at avbrytelsen da blir blokkert. Siden det ikke gis eksplisitt informasjon gjennom systemet om hva slags oppgave sykepleieren utfører, heller ikke informasjon om hva avbruddet gjelder, vil det ikke være aktuelt å kun tillate visse anrom basert på relevans i forhold til oppgave. Videre kan man si at muligheten til å sette status til utilgjengelig også vil ta i bruk teknikken fraråding, da symbolet som viser at en sykepleier er utilgjengelig og ønsker å ikke bli forstyrret implisitt fraråder andre sykepleiere å avbryte vedkomne. Her må hver enkelt selvfølgelig ta en vurdering på om innholdet i avbrytelsen er så viktig at det overgår den negative effekten ved avruddet. Endring i ringevolum fanges opp av teknikken modifisering, og reduserer interferensen mellom perseptuelle og kognitive prosesser til et minimum.
At sykeplierene vet hvilket rom som har utløst signalet, sammen med deres kunnskap om tilstanden til pasienten er viktig for hvordan de prioriterer signalet. Denne informasjonen fungerer da som en forhondsvisning. Selv om det i dette tilfellet ikke gir direkte informasjon angående hva signalet gjelder, vil romnummeret sammen med sykepleierens kunnskap om blandt annet pasientens tilastnad gi en indirekte indikasjon. Muligheten for å gå pasientene flere signalknapper for bruk ved forskjellige behov (som for eksempel drikke, toalettbesøk eller mer akutt hjelp), ble diskutert under workshopene, men deltagerene kom frem til at i tillegg til mulighet for misbruk fra pasientenes side, og faren for at sykepleierene skulle nedpriorietere signaler om mindre akutte behov, som drikke, er faren for at dersom en pasient skulle få et illebefinnende, og den første signalknappen denen fikk tak på var vannglasset kunne dette få fatale følger. Et annet forslag som kom frem på den andre dagen var et oversiktsbilde over rommet hvor signalet var utløst. Det var bred enighet om at dette ville være et godt grunnlag for å vurdere prioriteringen av signalet.   

\noindent
Etter introduksjon av prototypen, og muligheten til å sette sin status til utilgjengelig
for å formidle nettopp dette til kolleger så vi at dette var en funksjon som ble brukt
i større grad enn å gi muntlig beskjed. Det kan være flere grunner til dette, blandt annet at gjennom telefonen blir denne informasjonen distribuert til alle aktuelle sykepleiere, i stedet for bare dem som tilfeldigvis er tilstede der og da. Dette er også en mindre avbrytende måte å dele informasjon på, og er dermed med på å redusere mengden avbrytelser, som igjen kan hindre overbelastning av den enkeltes kognitive kapasitet. 

\noindent
Relasjonenen mellom sykepleieren som blir avbrutt av pasientsignalet og den pasienten som utløste signalet vil ha mye å si for hvordan sykepleieren prioriterer signalet. Dette er i tråd med Harr og Kaptelinins (2007) tanker om at oppførselen til avbryter og den avbrutte avhenger av mellommenskelige relasjoner. Selv om vi kan anta at pasientens adferd, eller valg om å avbrye eller ikke, ikke vil bli påvirket i nevneverdig grad i denne situasjonen, kom det tydelig frem under workshopene at dette for sykepleierene ofte er avgjørende for valget de tar i forhold til å besvare signalet eller ikke.  
Deres bekymring for hvordan den enkelte pasientent blir påvirket av at sykepleieren får et innkomne pasientsignal var tydelig. At også pasienten blir forstyrret når sykeplieren som er hos denne blir avbrutt, kalles av Harr og Kaptilinin (2007) for loksajonsbasert forstyrrelse. 