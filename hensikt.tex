\section{Hensikt}
\label{chp: hensikt}

Hensikten gjenspeiler den underliggende grunnen til å gjøre forskning, hva som gjør den interessant og hvorfor den er viktig eller nyttig. Videre kan man se på hvorfor man ønsker å forske på noe \cite{Oates}. 

\noindent
Motivasjonen for oppgaven var innledningsvis å avdekke fordeler og utfordringer i det eksisterende pasientsignalsystemet, og hvordan det kunne videreutvikles for å bedre støtte sykepleierenes behov. En gjennomgang av tidligere arbeid ga en bredere forståelse av forskningsområdet, og vi ønsket derfor å utdype problemstillingen. Vi formulerte to forskningsspørsmål som søkte svar på hvorvidt informasjon om kollegers aktiviteter og tilgjengelighet er nyttig, og hvordan slik informasjon kan kommuniseres på en hensiktsmessig måte. Da tidligere arbeid viste at avbrytelser kan ha en negativ effekt på sykepleieres arbeid, ønsket vi samtidig å undersøke hvordan systemet kan endres for å redusere eventuelle negative effekter.

\subsubsection{Forskningsspørsmål}
Vi har formulert to spørsmål som vi søker svar på gjennom denne oppgaven. Disse er som følger:

\begin{enumerate}
\item Er informasjon om kollegers aktiviteter og tilgjengelighet nyttig for sykepleiere, og hvordan kan slik informasjon distribueres på en hensiktsmessig måte? 
\item Hvordan kan systemet endres for å begrense potensielle negative effekter ved avbrytelser?
\end{enumerate}
