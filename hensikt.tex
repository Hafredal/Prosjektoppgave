\section{Hensikt}
\label{chp: hensikt}

Hensikten gjenspeiler naturligvis den underliggende grunnen til å gjøre forskning, hva gjør den interessant og hvorfor er den viktig eller nyttig. Videre kan man se på hvorfor man ønsker å forske på noe. Vårt utgangspunkt er gitt av oppgaveteksten, å designe et system som gjør det enklere for sykepleiere å kommunisere awareness-informasjon til hverandre på en hensiktsmessig måte. Tidligere arbeid har vist at avbrytelser kan ha en negativ effekt på sykepleieres arbeid. Vi ønsker derfor også å undersøke om disse avbrytelsene kan reduseres, og i det tilfellet de oppstår gjøres mindre forstyrrende. 

Motivasjonen for oppgaven er dermed å forstå sykepleiernes behov og utfordringer knyttet til systemet ved å gjøre en kvalitativ analyse av tidligere arbeid, utvikle en lavnivå-protoyp med de forbedringene og funksjonalitetene som analysen avdekker, og etter workshop med brukertesting evaluere løsningene. Avslutningsvis vil vi komme med videre forslag til funksjonalitet og forbedringer. Disse metodene vil grundigere gjennomgås under avsnittet om prosess. 