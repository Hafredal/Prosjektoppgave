\section{Hensikt}
\label{chp: hensikt}

Hensikten gjenspeiler naturligvis den underliggende grunnen til å gjøre forskning, hva som gjør den interessant og hvorfor den er viktig eller nyttig. Videre kan man se på hvorfor man ønsker å forske på noe. Vårt utgangspunkt er gitt av oppgaveteksten, å designe et system som gjør det enklere for sykepleiere å kommunisere awareness-informasjon til hverandre på en hensiktsmessig måte. Tidligere arbeid har vist at avbrytelser kan ha en negativ effekt på sykepleieres arbeid. Vi ønsker derfor også å undersøke om antallet slike avbrytelser kan reduseres, og i tilfeller de oppstår gjøres mindre forstyrrende. 

\noindent
Motivasjonen for oppgaven er dermed å forstå sykepleiernes behov og utfordringer knyttet til systemet ved å gjøre en kvalitativ analyse av tidligere arbeid, utvikle en lavnivå-protoyp med de forbedringene og funksjonaliteter som analysen avdekker behov for, og deretter gjennomføre workshops som datagrunnlag for evaluering av løsningen. Avslutningsvis vil vi komme med videre forslag til funksjonalitet og forbedringer. de overnevnte metodene vil grundigere gjennomgås under avsnittet om prosess. 

\subsection{Forskningsspørsmål}
Vi har formulert to spørsmål som vi søker svar på gjennom denne oppgaven. Disse er som følger:

\begin{enumerate}
\item Identifiser behov knyttet til funksjonalitet for støtte av distribusjon av \emph{awareness}-informasjon
\item Hvordan kan systemet endres for å begrense potensielle negative effekter ved avbrytelser?
\end{enumerate}
