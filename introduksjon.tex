\chapter{Introduksjon}
\label{chp: introduksjon}

Denne oppgaven omhandler systemet for pasientsignal ved St.Olavs Hospital, universitetssykehuset i Trondheim. 

\noindent
St. Olavs Hospital er universitetssykehus for 697 000 innbyggere i Midt-Norge, pr 1. januar 2013, og lokalsykehus for befolkningen i fylket med vel 300 000 innbyggere. I 2012 hadde sykehuset 131 547 innlagte pasienter og gjennomførte 554 083 polikliniske konsultasjoner \cite{stolavs}. Sykehuset har som mål i 2013 å legge til rette for innovasjon og å øke "[...] implementering av nye produkter, [...] og løsninger som bidrar til økt kvalitet, effektivitet, [...]"\ \cite{styring13}.

\noindent
Sykehuset bruker i dag et fast og et trådløst system for å varsle sykepleiere når pasienter trenger hjelp. Pasientene kan blant annet tilkalle sykepleier ved å trekke i snoren ved sengen. Sykepleieren med primæransvar for pasienten vil da bli oppringt på sin trådløse enhet. Dersom sykepleieren ikke har mulighet til å svare vil neste sykepleier bli oppringt, i henhold til bemanningsplanen. Denne løsningen gjør sykepleierene utsatt for eksterne avbrytelser i et allerede avbruddsdrevet miljø \cite{Klemets12}. Slike avbrudd i arbeidet kan ha både positive og negative effekter. Avbruddene kan eksempelvis gi informasjon som øker den enkeltes eller gruppens \emph{awareness}, som er nødvendig for koordinert og effektivt arbeid, og som gir bedre beslutningsgrunnlag for hvordan sykepleierene skal håndtere innkommende pasientsignaler. Slik systemet fungerer i dag er avbruddene avgjørende for å formidle denne informasjonen. Likevel vil disse i noen situasjoner være svært forstyrrende, da menneskehjernen kun har begrenset kognitiv kapasitet. Dersom denne overbelastes kan det hemme vedkommendes oppmerksomhet, noe som i verste fall kan føre til livstruende situasjoner. Ofte vil sykepleiere være involvert i aktiviteter hvor det er problematisk å svare på innkommende pasientsignaler, eksempelvis ved at de er ute til lunsj, befinner seg i isolasjonsrom, eller er hos en pasient de ikke ønsker å forlate. Likevel vil de vises som tilgjengelige i systemet, og dermed motta pasientsignaler som kan oppfattes som forstyrrende. Klemets, Evjemo og Kristiansen (2013) foreslår dermed at videre design av systemet bør tillate sykepleierene å sette seg selv som utilgjengelige. 

\noindent
Gjennom arbeidet med denne oppgaven har vi sett på om, og hvordan, det er mulig å øke sykepleierenes awareness om hverandres aktiviteter og tilgjengelighet, og om dette kan redusere eventuelle negative effekter ved innkomne pasientsignal. Dette er oppsummert i to forskningsspørsmål:

\begin{enumerate}
  \item Identifisere behov knyttet til funksjonalitet for støtte av distribusjon av awareness-informasjon.
  \item Hvordan kan systemet endres for å begrense potensielle negative effekter ved avbrytelser?
\end{enumerate}

\noindent
Disse to spørsmålene har vi forsøkt å finne svar på gjennom kvalitative litteraturstudier og workshops med utspillig av scenarioer, både med bruk av dagens system og der deltagerene fikk teste prototypen vi hadde designet. 

\subsubsection{Avgrensning og disposisjon}
Denne oppgaven er avgrenset til å omhandle pasientsignal og hvordan sykepleiere varsles om disse. Vi har ikke tatt for oss underliggende tekniske arkitekturer eller økonomiske aspekter. Vårt fokus er hvordan varslingen av sykepleiere, og informasjonen de har tilgjengelig, kan endres og utnyttes på en slik måte at de opprettholder tilstrekkelig awareness, samtidig som varslingene gir minst mulig negative effekter.

\noindent
Resten av oppgaven er strukturert som følger: Først, i kapittel \ref{chp:teori} blir relevant teori presentert, med fokus på kognitiv kapasitet, awareness og avbrudd. Kapittel \ref{chp:forskningsmetode} presenterer hensikten med oppgaven, samt metoder, strategier og prosesser vi har benyttet i vårt arbeid. Videre blir resultater fra litteraturstudier og gjennomførte workshops presentert i kapittel \ref{chp:resultater} og diskutert i kapittel \ref{chp:diskusjon}, før vi oppsummerer oppgaven og kommer med en konklusjon i kapittel \ref{chp:konklusjon}. I appendiksene til denne oppgaven er en beskrivelse av dagens pasientsignalsystem (Appendiks \ref{appendix_dagenssystem}), og beskrivelsen av gjennomføringen av workshops (Appendiks \ref{appendix_workshop}).