\chapter{Introduksjon}
\label{chp: introduksjon}

Denne oppgaven omhandler systemet for pasientsignal ved St.Olavs Hospital, universitetssykehuset i Trondheim. St. Olavs Hospital er universitetssykehus for Midt-Norge for 697 000 innbyggere pr 1. januar 2013, og lokalsykehus for befolkningen i fylket med vel 300 000 innbyggere. I 2012 hadde sykehuset 131 547 innlagte psienter og gjennomførte 554 083 polikliniske konsultasjoner. \cite{stolavs} Sykehuset har som mål i 2013 å legge til rette for innovasjon og å øke "[...] implementering av nye produkter, [...] løsninger som bidrar til økt kvalitet, effektivitet, [...]"\ \cite{styring13}.

\section{Bakgrunn for problemstilling}
Dagens sykehus bruker informasjonssystemer i stor grad for støtte i det daglige arbeidet. Det er en økende trend i å utstyre sykepleiere med trådløse enheter, for å øke deres tilgjengelighet og \emph{awareness} ovenfor pasientene og hverandre. Samtidig gjør dette dem mer utsatt for teknologi-baserte forstyrrelser og avbrudd i et allerede abruddsdrevet miljø \cite{Klemets12}. Denne balansen mellom awareness og avbrudd er hårfin, og på ingen måte svart/hvitt. Sykepleierenes awareness er nødvendig for koordinering av arbeid, men kun til en viss grad. Samtidig kan avbrudd være svært forsyrrende og avbryte livsviktig arbeid, men også bidra til nødvendig awareness. 

\noindent
Et eksempel på et slikt overnevnt informasjonsystem er det som gir pasientene mulighet til å gi beskjed om at de trenger hjelp, ved å trekke i snoren ved sengen. Denne muligheten har eksistert lenge, men systemet for vasling av sykepleierene er stadig under utvikling. I denne oppgaven har vi sett på dette systemet ved St.Olavs Hospital i Trondheim, og spørsmålene vi har forsøkt å svare på er som følger:

\begin{enumerate}
  \item Identifisere behov knyttet til funksjonalitet for støtte av distribusjon av awareness-informasjon.
  \item Hvordan kan systemet endres for å begrense potensielle negative effekter ved avbrytelser?
\end{enumerate}

\section{Avgrensning og disposisjon}
Denne oppgaven er avgrenset til å omhandle pasientsignal, og hvordan sykepleierene vasles om dette. Vi har ikke tatt for oss verken underliggende tekniske arkitekturer eller økonomiske aspekter ved dette. Vårt fokus er hvordan vaslingen av sykepleierene, og informasjonen de har tilgjengelig kan endres og utnyttes på en slik måte at de opprettholder tilstrekkelig awareness samtidig som vaslingene gir minst mulig negative avbrudd.

\noindent
Resten av denne oppgaven er strukturert som følger: Først, i kapittel \ref{chp:bakgrunn} beskrives bakgrunnen for oppgaven. I kapittel \ref{chp:teori} blir relevant teori presentert, med fokus på avbrudd og awareness. Kapittel \ref{chp:forskningsmetode} presenterer hensikten med oppgaven, og metoder, strategier og prosesser vi har brukt. Videre blir resultater fra gjennomførte workshops og en evaluering av disse presentert i [KAPITTEL], før vi oppsumerer oppgaven og kommer med en konklusjon i kapittel [KAPITTEL]. I appendiksene til denne oppgaven er en beskrivelse av dagens pasientsignalsystem (Appendiks \ref{appendix_dagenssystem}), og beskrivelsen av gjennomføringen av workshops (Appendiks \ref{appendix_workshop}).