\subsection{Suksesskriterier og utfordringer}
\label{chp: utfordringer}

En innledning : "Lorem ipsum dolor sit amet, consectetur adipisicing elit, sed do eiusmod tempor incididunt ut labore et dolore magna aliqua. Ut enim ad minim veniam, quis nostrud exercitation ullamco laboris nisi ut aliquip ex ea commodo consequat. Duis aute irure dolor in reprehenderit in voluptate velit esse cillum dolore eu fugiat nulla pariatur. Excepteur sint occaecat cupidatat non proident, sunt in culpa qui officia deserunt mollit anim id est laborum."

\subsubsection{Tre grunner til manglende suksess}
Mange CSCW-systemer faller for kort når det kommer til forventningene til deres suksess. Dette kommer gjerne til syne ved at de tiltenkte brukerene ganske enkelt unngår å bruke systemet, eller at de stadig må lage midlertidige løsninger (workarounds) for å få gejnnomført arbeidsoppgavene sine.

Disse systemene deler gjerne én eller flere grunner til dette.

For det første er det ofte et misforhold mellom de som tjener på implementeringen av systemet og brukerene. De som tjener på systemet er gjerne ledere som har avgjort at systemet skal implementeres, mens det skaper merarbeid arbeidstakerene som blir pålagt å bruke systemet, uten å gi dem noen form for fordeler. Dette merarbeidet kan bland annet være i form av mer dokumentering og registrering.

For det andre er det ofte manglede kunnskap om CSCW (multibruker-)systemer, i motsetning til enkeltbruker-systemer. En typisk CSCW-applikasjon eller system vil vanligvis bli brukt av et vidt spekter av forskjellige brukere, med forskjellig bakgrunn, erfaringer og forhold til bruk av informasjonssystemer generelt. Beslutningstakere, ofte ledere, vil ha en intuisjon for hva som vil være fordelaktige for brukere som en selv. Desverre kan de fort overse funksjoner som andre brukere vil ha nytte av, spesielt for fuksjoner som vil før til merarbeid for dem selv. 

For det tredje er det vanskelig å lære fra tidligere feil, da CSCW-systemer er svært komplekse og unike for hvert enkelt tilfelle, noe som vanskeliggjør evaluering i ettertid. Det er også vanskelig, om ikke umulig å gjenskape miljøene og forholdene som er essensielle i den virekelige konteksten hvor et CSCW-system skal implementeres i et laboratorium. for ikke å snakke om utfordringene ved å gjøre feltobservasjoner, grunnet blandt annet variasjon i gruppesammensettning og miljømessige faktorer. \cite{Grundin88}

\subsubsection{[ARBEIDSTITTEL]}
Det å utvikle en CSCW-applikasjon helseomsorgen vil aldri bli en enkel prosess. Det er en konflikt mellom det flytende samarbeidet og tilsynelatende men nødvendig rotete arbeismåten til sykepleiere og den formelle, standardiserte og relativt stive funksjonaliteten til et informasjonssystem. Derfor er en av forutsettningene for et suksessfullt system i et slikt miljø å ikke forsøke å erstatte denne 'rotetheten' med en strømlinjeformet og rasjonalitet som ofte er vanlig for slike systemer. Verktøy som kun har forutbestemte sekvensiell trinn, eller som tillater kun gitte typer data-input vil derfor ikke fungere sammen med måten sykepleierene arbeider på, og som en følge av dette ikke overleve.\cite{Berg99}

\noindent
