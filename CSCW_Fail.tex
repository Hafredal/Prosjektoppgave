\subsection{Utfordringer med CSCW}
\label{chp: utfordringerMedCSCW}

Mange CSCW-systemer faller til kort når det kommer til forventningene til deres suksess. Dette kommer gjerne til syne ved at de tiltenkte brukerene ganske enkelt unngår å bruke systemet, eller at de stadig må lage midlertidige løsninger (workarounds) for å få gejnnomført arbeidsoppgavene sine. Spesielt to grunner til dette er knyttet til kompleksiteten rundt å utvikle multibrukersystemer, som brukes samtidig av mange brukere på forskjellige nivåer og med forskjellige behov og perspektiver.

\noindent
For det første er det ofte manglede kunnskap om CSCW (multibruker-)systemer, i motsetning til enkeltbruker-systemer. En typisk CSCW-applikasjon eller system vil vanligvis bli brukt av et vidt spekter av forskjellige brukere, med forskjellig bakgrunn, erfaringer og forhold til bruk av informasjonssystemer generelt. Beslutningstakere, ofte ledere, vil ha en intuisjon for hva som vil være fordelaktige for brukere som en selv. Desverre kan de fort overse funksjoner som andre brukere vil ha nytte av, spesielt for fuksjoner som vil før til merarbeid for dem selv. 

\noindent
Det andre er at det vanskelig å lære fra tidligere feil, da CSCW-systemer er svært komplekse og unike for hvert enkelt tilfelle, noe som vanskeliggjør evaluering i ettertid. Det er også vanskelig, om ikke umulig å gjenskape miljøene og forholdene som er essensielle i den virekelige konteksten hvor et CSCW-system skal implementeres i et laboratorium. for ikke å snakke om utfordringene ved å gjøre feltobservasjoner, grunnet blandt annet variasjon i gruppesammensettning og miljømessige faktorer.

