\subsection{Utfordringer med CSCW}
\label{chp: utfordringerMedCSCW}

Mange CSCW-systemer faller til kort når det kommer til forventningene til deres suksess. Dette kommer gjerne til syne ved at de tiltenkte brukerene ganske enkelt unngår å bruke systemet, eller at de stadig må lage midlertidige løsninger (workarounds) for å få gjennomført arbeidsoppgavene sine. Det er spesielt to grunner til dette, knyttet til kompleksiteten rundt å utvikle multibrukersystemer, som brukes samtidig av mange brukere på forskjellige nivåer og med forskjellige behov og perspektiver.

\noindent
For det første er det ofte manglede kunnskap om CSCW (multibruker-)systemer, i motsetning til enkeltbruker-systemer. En typisk CSCW-applikasjon eller system vil vanligvis bli brukt av et vidt spekter av forskjellige brukere, med forskjellig bakgrunn, erfaringer og forhold til bruk av informasjonssystemer generelt. Beslutningstakere, ofte ledere, vil ha en intuisjon for hva som vil være fordelaktig for brukere som dem selv. Desverre kan de fort overse funksjoner som andre brukere vil ha nytte av, spesielt for fuksjoner som vil før til merarbeid for dem selv. 

\noindent
For det andre er det vanskelig å lære fra tidligere feil, da CSCW-systemer er svært komplekse og unike for hvert enkelt tilfelle, noe som vanskeliggjør evaluering i ettertid. Det er også vanskelig, om ikke umulig, å gjenskape miljøene og forholdene som er essensielle i den virekelige konteksten hvor et CSCW-system skal implementeres i et laboratorium. For ikke å snakke om utfordringene ved å gjøre feltobservasjoner, grunnet blandt annet variasjon i gruppesammensettning og miljømessige faktorer.

\noindent
Utifra de to overstående punktene ser vi at det å utvikle en CSCW-applikasjon helseomsorgen vil aldri bli en enkel prosess. Det er en konflikt mellom det flytende samarbeidet og tilsynelatende men nødvendig rotete arbeidsmåten til sykepleiere og den formelle, standardiserte og relativt stive funksjonaliteten til et informasjonsystem. Derfor er en av forutsettningene for et suksessfullt system i et slikt miljø å ikke forsøke å erstatte denne 'rotetheten' med en strømlinjeformet og rasjonalitet som ofte er vanlig for slike systemer. Verktøy som kun har forutbestemte sekvensiell trinn, eller som tillater kun gitte typer data-input vil derfor ikke fungere sammen med måten sykepleierene arbeider på, og som en følge av dette ikke overleve.\cite{Berg99}