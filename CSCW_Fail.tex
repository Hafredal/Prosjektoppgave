\subsection{Utfordringer med CSCW}
\label{chp: utfordringerMedCSCW}

Ifølge Berg (1999) vil mange CSCW-systemer komme til kort i forhold til forventningene til deres suksess. Dette kommer gjerne til syne ved at de tiltenkte brukerene unngår å bruke systemet, eller at de stadig må lage midlertidige løsninger (workarounds) for å få gjennomført arbeidsoppgavene sine. 
Kompleksiteten ved det å utvikle multibrukersystemer oppstår da de brukes samtidig av mange brukere på forskjellige nivåer, med forskjellige behov og perspektiver.

\noindent 
Et typisk CSCW-system vil vanligvis bli brukt av et vidt spekter av brukere, med forskjellig bakgrunn, erfaringer og forhold til bruk av informasjonssystemer generelt. Det oppstår dermed en risiko for at beslutningstakere tar avgjørelser basert på hva som vil være fordelaktig for brukere som dem selv, og dermed overser funksjoner som andre brukere kan ha nytte av. Funksjonalitet som gjør noe lettere for en gruppe brukere, kan gi merarbeid til en annen gruppe brukere. 
\noindent
Det kan også være vanskelig å lære fra tidligere feil, da CSCW-systemer er svært komplekse og unike for hvert enkelt tilfelle, noe som vanskeliggjør evaluering i ettertid. Det er utfordrende å gjenskape miljøet og forholdene som er essensielle i den virkelige konteksten hvor et CSCW-system skal implementeres i et laboratorium. Feltobservasjoner kan gi et feilaktig inntrykk, og kan variere avhengig av gruppesammensetting og miljømessige faktorer.

\noindent
Det å utvikle et CSCW-system for helseomsorgen vil dermed kunne være en utfordrende prosess. Det er en konflikt mellom det flytende samarbeidet og de tilsynelatende uforutsette arbeidsoppgavene til sykepleiere, og den formelle, standardiserte og relativt stive funksjonaliteten til et informasjonsystem. Derfor er en av forutsetningene for et suksessfullt system i et slikt miljø, å ikke forsøke å erstatte denne 'rotetheten' med en rasjonalitet og strømlinjeform som ofte er vanlig for slike systemer. Verktøy som kun har forutbestemte sekvensielle trinn, eller som kun tillater gitte typer input vil derfor ikke fungere sammen med måten sykepleierene arbeider på, og som en følge av dette ikke overleve \cite{Berg99}.