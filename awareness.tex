\section{Awareness}
\label{chp: awareness}

Sykepleiere må ha et svært godt samarbeid seg i mellom for å kunne yte best mulig pasientomsorg. For å oppnå dette er de avhengige av at alle til en hver tid har oversikt over situasjonen til resten av sykpleierene på teamet \cite{Evjemo}. Denne typen oversikt kalles gjerne for bevissthet, og er et konsept med stadig større betydning, spesielt med tanke på CSCW \cite{Dourish92}. En slik bevissthet om andres aktiviteter er avgjørene for å sikre et vellykket samarbeid og god koordinasjon\cite{KlemetsRedundancy}. 

\noindent
\emph{Definisjoner og konseptualiseringer}

\noindent
Bevissthet er et svært uklart og komplekst konsept med mange definisjoner og konseptualiseringer \cite{KlemetsRedundancy}\cite{Gutwin04}\cite{Schmidt02}. Begrepet bevissthet brukes i stadig flere sammenhenger, og mange forskere blir stedig mindre komfortable med å bruke begrepet i seg selv. For å komme til en slags konseptuell klarhet i hva bevissthet skal bety må vi innse at det ikke gir mening å se på bevissthet noe i seg selv, men må referere til en persons bevissthet \emph{om} noe \cite{Schmidt02}. Dermed har vi fått konsepter som blandt annet sosial eller kontekstformidlet sosial bevissthet \cite{Bardram04}, generell bevissthet\cite{Gross13}, gruppe-bevissthet\cite{Gutwin04}, perifer-, bakgrunns-, passiv- og gjensidig bevissthet\cite{Schmidt02}, noe som også gir mange forskjellige definisjoner. Dourish og Bellotti har definert konseptet som \emph{"...en forståelse av aktivitetene til andre, som gir kontekst for dine egne aktiviteter"}. En mer inngående definisjon finner vi i \cite{Endsly95}, som tar for seg konseptualliseringen situasjonell bevissthet: \emph{"Situasjonell awareness er oppfattelse av elemntene i omgivelsene avgrenset av tid og rom, forståelsen av ders mening, og prognosen for deres status i nær fremtid"}. i \cite{Gross13} er generell bevissthet definert som "den gjennomgripende opplevelsen av å vite hvem som befinner seg i nærheten, hva de driver med, om de er relativt opptatt eller kan engasjeres osv". Selv om begrepet bevistthet, som vi ser, ikke har en entydig definisjon er det viktig å legge merke til at bevissthet er en persons \emph{indre} kunnskap og forståelse av situasjonen.\cite{Gross13} 

\noindent
I følge C. Heath et al., oppnås bevissthet gjennom kontinuerlig interaksjon med andre, og er på ingen måte en sinnstilstand. Dette underbygger tankene bak konseptualiseringen tillstandsbevissthet. Her kreves et visst nivå av kognitive ferdigheter for å til en hver tid kunne overvåke og oppdage endringer i miljøet en er i. Dette skjer samtidig som man synliggjør for omgivelsene ens egen nåværende situasjon gjennom implisitte eller eksplisitte signaler. Med informasjonen man henter inn vil man kunne danne seg et mentalt bilde av hvordan situasjonen til enhver tid er \cite{Bardram04}. 
