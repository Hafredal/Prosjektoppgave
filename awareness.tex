\section{Awareness}
\label{chp: awareness}

Sykepleiere må ha et svært godt samarbeid seg i mellom for å kunne yte best mulig pasientomsorg. For å oppnå dette er de avhengige av at alle til en hver tid har oversikt over situasjonen til resten av sykpleierene på teamet \cite{Evjemo}. Denne typen oversikt kalles gjerne for bevissthet, og er et konsept med stadig større betydning, spesielt med tanke på CSCW \cite{Dourish92}. En slik bevissthet om andres aktiviteter er avgjørene for å sikre et vellykket samarbeid og god koordinasjon\cite{KlemetsRedundancy}. Desverre har de ikke alltid mulig å oppdatere hverandre ansikt-til-ansikt. Et eksempel som kan trekkes frem er infeksjonkritiske miljøer, som isolasjonsrom. Her er det ikke mulig å bruke telefon eller gå ut av rommet uten å gå gjennom en ny injeksjonsprosess, noe som understreker behovet for et godt tilpasset system for å  øke sykepleierenes bevissthet om hverandre.

\noindent
Bevissthet er et komplekst konseptet \cite{KlemetsRedundancy}\cite{Gutwin04} med mange konseptualiseringer og definisjoner. Dourish og Bellotti har definert konseptet som \emph{"...en forståelse av aktivitetene til andre, som gir kontekst for dine egne aktiviteter"}. Bevissthet oppnås, i følge C. Heath et al., gjennom kontinuerlig interaksjon med andre, og er på ingen måte en sinnstilstand. Dette underbygger tankene bak konseptualiseringen tillstandsbevissthet. Her kreves et visst nivå av kognitive ferdigheter for å til en hver tid kunne overvåke og oppdage endringer i miljøet en er i. Dette skjer samtidig som man synliggjør for omgivelsene ens egen nåværende situasjon gjennom implisitte eller eksplisitte signaler. Med informasjonen man henter inn vil man kunne danne seg et mentalt bilde av hvordan situasjonen til enhver tid er \cite{Evjemo}. 
