\section{Awareness}
\label{chp: awareness}

Et sykehus er en organisasjon hvor vi finner mange informasjonskilder, plutselig stress, rutinearbeid og ansatte som stadig er på forskjellige steder. Dette gjør at kommunikasjon er svært viktig for å sikre god pasientomsorg og -sikkerhet\cite{Klemets12}. Uttrykket funksjonsredundans (av det engelske 'redundancy of function') betyr at kunnskap om situasjonen til en hver tid er delt, og overlappende mellom medlemmene i gruppen.\cite{KlemetsRedundancy} Denne typen kunnskap kalles gjerne for bevissthet, og er et konsept med stadig større betydning, spesielt med tanke på CSCW \cite{Dourish92}. En slik bevissthet om andres aktiviteter er avgjørene for å sikre et vellykket samarbeid og god koordinasjon\cite{KlemetsRedundancy}. 

\noindent
\emph{Definisjoner og konseptualiseringer}

\noindent
Bevissthet er et svært uklart og komplekst konsept med mange definisjoner og konseptualiseringer \cite{KlemetsRedundancy}\cite{Gutwin04}\cite{Schmidt02}. Begrepet bevissthet brukes i stadig flere sammenhenger, og mange forskere blir stedig mindre komfortable med å bruke begrepet i seg selv. For å komme til en slags konseptuell klarhet i hva bevissthet skal bety må vi innse at det ikke gir mening å se på bevissthet noe i seg selv, men må referere til en persons bevissthet \emph{om} noe \cite{Schmidt02}. Dermed har vi fått konsepter som blandt annet sosial, eller kontekstformidlet sosial bevissthet \cite{Bardram04}, generell bevissthet\cite{Gross13}, gruppe-bevissthet\cite{Gutwin04}, perifer-, bakgrunns-, passiv- og gjensidig bevissthet\cite{Schmidt02}, noe som også gir mange forskjellige definisjoner. Dourish og Bellotti har definert begrepet som \emph{"...en forståelse av aktivitetene til andre, som gir kontekst for dine egne aktiviteter"}. En mer inngående definisjon finner vi i \cite{Endsly95}, som tar for seg konseptualliseringen situasjonell bevissthet: \emph{"Situasjonell bevissthet er oppfattelse av elemntene i omgivelsene avgrenset av tid og rom, forståelsen av deres mening, og prognosen for deres status i nær fremtid"}. i \cite{Gross13} er generell bevissthet definert som "den gjennomgripende opplevelsen av å vite hvem som befinner seg i nærheten, hva de driver med, om de er relativt opptatt eller kan engasjeres osv". Bardrams konsept kontekstformidlet sosial bevissthet har som mål å minimere uønskede forstyrrelser mellom mobile kolleger ved å øke deres sosiale bevissthet ved hjelp av kontekstbevisste systemer\cite{Bardram04}. Selv om begrepet bevissthet, som vi ser, ikke har en entydig definisjon er det viktig å legge merke til at bevissthet er en persons \emph{indre} kunnskap og forståelse av situasjonen.\cite{Gross13} 

\noindent
I følge både C. Heath et al. (2002) og Schmidt (2002), oppnås bevissthet gjennom kontinuerlig interaksjon med andre, og er på ingen måte en sinnstilstand eller passiv aktivitet. Dette underbygger tankene bak konseptualiseringen tillstandsbevissthet. Her kreves et visst nivå av kognitive ferdigheter for å til en hver tid kunne overvåke og oppdage endringer i omgivelsene. Dette skjer samtidig som man synliggjør for omgivelsene ens egen nåværende situasjon gjennom implisitte eller eksplisitte signaler. Med informasjonen man henter inn vil man da kunne danne seg et mentalt bilde av hvordan situasjonen til enhver tid er. I settinger hvor man ikke alltid er samlokalisert, som på et sykehus, kan bruk av gjenstander som whiteboards og skjermer være hensiktsmessig for å støtte opp under deling og innhenting av overnevnte informasjon\cite{Bardram04}. 

\noindent
Selv om god pasientomsorg er avhengig av sykepleierenes oversikt over situasjonen til en hver tid må det også tas hensyn til pasienters og pårørenes privatliv. Får å opprettholde en god balanse mellom disse stilles det høye krav til sykepleierenes evne til å tilegne seg kunnskap om situasjonen på en måte som i minst mulig grad virker påtrengende for pasienter å pårørende, og at informasjon distribuert gjennom skjermer og tavler synlig for andre enn kun helsepersonell ikke krenker pasientenes privatliv\cite{Ebright10}

\noindent
I denne oppgaven velger vi å bruke blablabla sin definisjon fordi blablabla....
