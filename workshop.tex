\section{Workshopene}
\label{ws}
I denne delen vil vi legge frem funn vi gjorde i løpet av WS1 og WS2. Datamaterialet ble analysert ved bruk av stegvis-deduktiv induktiv metode, og resultatene som presenteres er de av relevans til våre forskningsspørsmål. Betegnelsene sykepleier, student, deltager og sykepleierstudent referer til sykepleierstudentene som deltok i workshopene. Begrepene signal og pasientsignal brukes om hverandre.

\subsection{Oversikt}
Det var ønskelig å avklare i hvilken grad sykepleierne har oversikt over inneliggende pasienter på sitt sengetun, og over kollegers ansvar og aktiviteter.

\noindent
Under vaktskiftemøtet blir det gitt detaljert informasjon om alle pasientene på sengetunet, og sykepleierne fordeler ansvaret for disse. Flere av deltagerne forteller at de har med seg en skriftlig oversikt over pasientene, hvor detaljer fra vaktskiftemøtet noteres. S1-A1 fortalte at hun bruker denne listen hvis hun skal inn til en pasient hun ikke har vært hos før. Denne oversikten opprettholdes med mindre uforutsette hendelser oppstår, eksempelvis at pasienter bytter rom eller det kommer noen som må ha øyeblikkelig hjelp. Det kom også frem at det etterstrebes at en pasient har den samme primærsykepleieren over tid, så godt det lar seg gjøre.

\noindent
S1-A1 poengterte at selv om sykepleierne har oversikt over kollegers ansvar kan de ikke vite hvor alle er til enhver tid. Derimot vil kunnskapen om kollegenes ansvar gi en pekepinn på hvor det er sannsynlig å finne dem. Likevel er det utbredt bruk av telefon for å få tak i hverandre. Flere av studentene sa at de ofte glemmer å markere seg som tilstede når de går inn på et rom. S1-A1 fortalte at de ikke bruker telefon for mottak av pasientsignaler, og om de da glemmer å markere seg som tilstede vil de ikke få noen indikasjon på at signaler er utløst. S3-A3 mente derfor at implementering av automatisk tilstedemarkering vil være en stor forbedring. 

\noindent
Hvorvidt det er vanlig å si ifra til kolleger når man går noe sted, for eksempel inn til en pasient, til et annet tun, avdeling eller etasje, eller til lunch viste seg å være svært variabelt. Dette var tydelig under utspillingen av scenarioene uten prototype, hvor kun én av fire varslet sykepleieren tilstede på tunet om at hun skulle inn på et pasientrom. På spørsmål om man ville varslet andre før man går inn på et rom med en urolig pasient (noe som kan ta lengre tid) svarte S5-A3: \emph{$"$ja, man ville jo egentlig sagt det, men det er jo sjelden det blir sagt da, for de andre er jo gjerne klar over hva som skjer$"$}. Dette kan tyde på at sykepleierne regner med at de andre har oversikt over hvilke pasienter som ligger på hvilke rom, og hvilke aktiviteter som skal gjøres der.

\noindent
Etter at prototypen og dens funksjon for å sette status til $"$utilgjengelig$"$ ble introdusert, så vi at dette ble gjort i større grad enn å gi muntlig beskjed. S1-A1 og S6-A4, som begge skulle utføre sårskift, satte seg som $"$utilgjengelig$"$ før de gikk inn til pasienten. S6-A4 spurte sykepleieren på tunet om det var greit at hun satte seg som $"$utilgjengelig$"$ da hun ville være opptatt en stund, og om kollegaen derfor kunne besvare signaler fra hennes pasienter. Da S2-A2, som skulle inn til en urolig pasient, ble spurt om hun ville sagt ifra til noen før hun gikk inn, svarte hun at: \emph{$"$hvis appen skal funke, burde man vel ikke trenge det$"$}.

\noindent
Deltagerne kommenterte at var viktig at muligheten til å sette seg som $"$utilgjengelig$"$ ikke blir misbrukt, og S6-A4 var redd for at hun selv skulle glemme å manuelt sette seg til $"$tilgjengelig$"$ igjen. For å forhindre at sykepleierne blir stående som $"$utilgjengelig$"$ når de ikke er det, ble det foreslått flere mulige løsninger av både forskere og deltagere: (1) At systemet automatisk endrer status til $"$tilgjengelig$"$ når sykepleieren forlater rommet, (2) at telefonen gir en påminnelse om at sykepleieren fremdeles står som $"$utilgjengelig$"$ etter en viss tid, eller (3) at det er en tidsbegrensing på hvor lenge sykepleierne kan stå som $"$utilgjengelig$"$. Disse ble diskutert uten at det ble konkludert med en foretrukket løsning. 

\subsection{Håndtering av innkommende pasientsignal}
Under workshopene kom det frem at sykepleierne tar hensyn til flere faktorer når de vurderer og håndterer innkommende pasientsignaler. 

\noindent
Alle deltagerne var enige om at dersom det var en pasient de selv hadde primæransvar for, ville de såfremt de hadde mulighet gå til pasienten med en gang. Derimot fortalte studentene at de ved signal fra en pasient de ikke er primær for, lar telefonen ringe en stund for å se om noen andre (hovedsaklig primærsykepleier) svarer. Dette til hensyn for pasienten, slik at denne skal slippe å forholde seg til flere sykepleiere enn nødvendig. I tilfeller hvor pasientsignalet ikke besvares, ville noen av deltagerne gått inn til pasienten, mens andre først ville hørt med eventuelle kolleger i nærheten om noen av de har primæransvar for pasienten. Som det ble poengtert av S1-A1 vil ikke fordelingen av pasienter mellom sykepleierne ha noen hensikt dersom alle svarer på alle signaler. Hvis sykepleierne derimot vet at en pasients tilstand er dårlig, vil de i større grad prioritere å se til pasienten fremfor å holde fast ved ansvarsfordelingen. 

\noindent
Studentene kom med flere forslag til hvilken informasjon de kan tenke seg å motta på telefonen, for å være i bedre stand til å vurdere hvordan de skal håndtere det innkommende signalet. S3-A3 uttrykte at visningen av de andre sykepleiernes tilgjengelighet ved innkommende pasientsignal virket som en god idé, \emph{$"$for da ser du om det er noen som kan ta den, eller om du må vurdere å gjøre det selv$"$}. Det var derimot ingen av deltagerne som benyttet seg av denne informasjonen da de utspilte scenarioene med prototype. S4-A3 påpekte at det kan være nyttig å få en kort forklaring av hva som feiler pasienten, da man vil slippe å se på listen man har med seg. Et annet forslag var å oppgi hvor i rommet signalet er utløst (ved sengen, ved døren, på toalettet og lignende), da dette kan gi en indikasjon på hva signalet gjelder. Studentene påpekte at man aldri vet hvorfor pasienten utløser et signal, det kan være at pasienten vil ha et glass vann, eller noe mer alvorlig. Det eneste de vet er at noen på rommet trenger assistanse. For å gi sykepleierne mer informasjon om hva signalet gjelder ble det derfor foreslått å ha forskjellige signalknapper for pasientenes ulike behov, eksempelvis toalett, drikke, eller annen assistanse. Deltagerne kom  frem til at dette i mange tilfeller kunne gi god støtte for prioritering dersom man allerede var opptatt, men at det imidlertid ikke var en ønsket funksjon av flere grunner. For det første kan noen av pasientene som ligger inne være svært syke, og kan ha problemer med å velge riktig knapp. Dersom en pasient skulle få et illebefinnende og den eneste knappen vedkommende når er signalet for drikke, kan dette i verste fall få fatale følger. Studentene var usikre på om mindre akutte signaler, eksempelvis ønske om drikke, kan bli nedprioritert av sykepleierne, og at pasientene vil  vente unødvendig lenge på hjelp. De var også redde for at pasientene vil få en høyere terskel for å tilkalle hjelp, hvis de for eksempel $"$bare vil ha et glass vann$"$. Disse ulike signalene kan også bli misbrukt dersom pasienter trykker på en knapp som signaliserer et mer akutt behov enn det som er tilfellet. Et annet forslag var at sykepleierne mottar et oversiktsbilde av rommet hvor signalet er utløst på telefonen, slik at de får et inntrykk av pasientens tilstand og hvor i rommet denne befinner seg. Dersom sykepleieren er inne hos en annen pasient ved utløst signal, vil denne pasientens tilstand også tas i betraktning. 

\subsection{Varsling fra telefon og veggpanel}
Det var interessant å finne ut hvorvidt det innkommende pasientsignalet påvirker sykepleiernes arbeid, og i så fall hvordan. Det kom tydelig frem under utspillingen av scenarioene at dette varierte veldig. Noen uttrykte at de ble stresset, mens andre sa at det ikke plaget dem. S4-A3 tenkte mest på hvordan det påvirket pasienten hun var hos, mens S9-A3 savnet å få en bekreftelse på at noen faktisk hadde gått inn til pasienten som utløste signalet, slik at hun kunne slutte å tenke på det. 

\noindent
Det ble ved flere anledninger under WS1 og WS2 påpekt at det i visse situasjoner ville være positivt om telefonen ikke ringte like høyt, eventuelt bare vibrerte. Dette var i stor grad av hensyn til pasientene, da flere av studentene opplever at pasienter blir skremt av at det ringer, eller føler seg til bry. S2-A2 mente det er feil at pasientene må vise forståelse og godta at andre signaler forstyrrer dem. Hun mente også det er viktig at sykepleierne tar ansvar for at dette blir håndtert på en god måte slik at pasientene ikke føler de er til bry. 

\noindent
Under scenarioet med sårskift valgte S1-A1 å sette seg til $"$utilgjengelig$"$. Da det tidligere hadde blitt foreslått at telefonene ikke skulle gi noen form for varsling ved en slik status, fikk hun derfor beskjed av fasilitator om at det var utløst et signal fra et annet rom som hun av den grunn ikke ble varslet om. Dette førte til at deltagerne under fokusgruppediskusjonen i etterkant i stor grad ble enige om at dette var lite hensikstmessig, da de faktisk ønsket å ha en viss oversikt over hva som skjedde. I forbindelse med denne diskusjonen kom det også forslag om å flytte veggpanelet fra døren til veggen over sengen. Dette gjelder spesielt dersom signalet blir varslet med svakere, eller ingen lyd, slik at sykepleierne likevel ikke går glipp av innkommende pasientsignaler.

\noindent
Etter introduksjon av prototypen ble det diskutert hvordan sykepleiernes status vil påvirke hvem som mottar pasientsignaler, og det kom flere forslag til hvordan dette kan løses. Det ble blant annet foreslått at signalet først sendes til de som er $"$tilgjengelige$"$, deretter de som er $"$opptatte$"$, og til slutt til de som er $"$utilgjengelige$"$. S2-A2 foreslo at de som er $"$opptatte$"$ eller $"$utilgjengelige$"$ først burde varsles med kun vibrasjon. Dersom signalet ble sendt til alle sykepleierne uten at noen gikk inn til pasienten, burde derimot alle varsles med lyd i neste runde. S1-A1 påpekte at hun så absolutt ville ha tydelig varsling dersom en av pasientene hun har ansvar for har utløst et signal som har gått lenge uten å bli besvart. Hun mente det er viktig å være sikker på at pasienten får hjelp innen rimelig tid. S2-A2 påpekte at det kan være et problem dersom de som står som $"$utilgjengelige$"$ ikke mottar signaler i det hele tatt,  og at det må finnes en mekanisme som forhindrer at alle sykepleierne på vakt er $"$utilgjengelige$"$ på samme tid. 