\section{Workshopene}
\label{ws}

\subsection{Funn før introduksjon av prototype}
Disse funnene er knyttet til dagens system, hvordan det fungerer, hva som fungerer bra og hvilke utfordringer studentene opplever. Disse viser behov for endringer av systemet, både med tanke på hvilken informasjon som formidles ved utløst signal, og hvordan dette formidles.

\subsubsection{Oversikt over pasienter}
Det var innledningsvis ønskelig å avklare i hvilken grad sykepleierene har oversikt over inneliggende pasienter på sitt sengetun. Studentene fortalte at det blir gitt informasjon om pasientene til de som går på vakt på vaktskiftemøtet .

\noindent
S3-A3: \emph{"Det er ganske nøye på rapportene, så du vet hvem alle er og hva det dreier seg om, forandringer fra dag til dag, selv om du ikke har vært inne hos dem."}

\noindent
Flere av deltagerene har en liste over inneliggende pasienter, hvor detaljer fra vaktskiftemøtet noteres.

\noindent
S1-A1: \emph{"Ja så hvis man skal inn til en pasient, hvis jeg ikke har vært der før, så tar jeg opp lappen."}

\noindent
S1-A1 sa videre at denne oversikten opprettholdes, med mindre uforutsette hendelser oppstår, eksempelvis at pasienter bytter rom eller det kommer noen som må ha øyeblikkelig hjelp.
 
 
\subsubsection{Oversikt over kolleger}
Sykepleierenes oversikt over kollegers ansvar og aktiviteter er av stor relevans til vår oppgave, og vi ønsket derfor å undersøke om sykepleierene informerer hverandre om sine egne aktiviteter.

\noindent
S1-A1: \emph{"Man kan vel ikke vite hvor alle er hele tiden, men man vet hvor de har ansvar. Så hvis man skal ha tak i noen, ringer man på telefonen, eller ser at vedkommende har ansvar for rom 1 og 2, så da sjekker man der. Men vi har ikke noe peiler på hvor alle er til enhver tid nei."} 

\noindent
Under fokusgruppediskusjonen, kom det tydelig frem at dette varierer i stor grad.

\noindent
J: \emph{"Men sier dere ifra når man går noe sted."}

\noindent
S1-A1: \emph{"Hvis man bare skal opp i fjerdeetasje eller andreetasje for å hente noe, tror jeg de bare går. Det er det inntrykket jeg har fra min avdeling hvertfall."} 

\noindent
S3-A3: \emph{"De sier jo i fra."}

\noindent
S4-A3: \emph{"Ikke hele tiden men, litt begge deler."}

\noindent
Dette var også tydelig under utspillingen av scenarioene uten prototype, hvor kun én av fire varslet sykepleieren tilstede på tunet om at hun skulle inn på et pasientrom. På spørsmål om man ville varslet andre før man går inn på et rom med en urolig pasient svarte S5-A3: \emph{"ja, man ville jo egentlig sagt det, men det er jo sjelden det blir sagt da, for de andre er jo gjerne klare over hva som skjer."} 

\subsubsection{Håndtering av innkommende pasientsignal}
Det kom frem under workshopene at sykepleierene tar hensyn til flere faktorer når de vurderer og håndterer innkommende pasientsignaler. 
\begin{itemize}
\item Flere av studentene fortalte at de ved anrop fra en pasient de ikke er primær for, lar telefonen ringe en stund for å se om andre svarer. Dette til hensyn for pasienten, slik at denne skal slippe å forholde seg til flere sykepleiere og for å holde en oversikt, som S1-A1 kommenterte: \emph{"ellers kan jo alle bare gå inn overalt, og da er det ikke noe vits i den fordelingen av pasienter lenger."} Hvis derimot et signal ikke besvares, poengterte S6-A4: \emph{"hvis det ringer på et rom og jeg spør rundt meg "er det noen av dere som har 203?" og ingen reagerer, så går jeg - folk kan jo være på toalettet eller noe"}  
\item Under utspillingen av scenario uten prototype, hvor sykepleieren var inne hos Jenny og det ble utløst et pasientsignal fra Adams rom, sa S4-A3: \emph{"først ville jeg avvist signalet for å se om noen andre ville tatt den, når jeg først står her inne og snakker."} Da de andre deltagerne ble spurt om hvordan de ville håndert en slik situasjon, svarte S2-A2: \emph{"du har jo litt overblikk over hva som feiler de andre pasientene, om noen ha veldig dårlig pust eller veldig dårlig rytme, og du ser at den ringer på, så må du gå og sjekke."}  
\item På spørsmål om sykepleieren ville hatt mer informasjon om pasienten som har utløst signalet, svarte S7-A1: \emph{"vet ikke, det er jo alltid noen rom som ringer ofte, og om ting som ikke haster, men man skal jo egentlig gå å se. Man kan jo ikke regne med at han bare skal ha et glass vann."} S9-A3 sa at hun kunne tenke seg mer detaljert informasjon om hvor signalet ble utløst fra, om det var fra sengen, badet eller ved døren, og S5-A3 foreslo at man kunne motta et bilde av rommet på telefonen.
\end{itemize}

\subsubsection{Varsling fra telefon og veggpanel}
For hvert utspilt scenario ønsket vi å finne ut hvordan det innkommende pasientsignalet påvirket sykepleierenes arbeid. Funnene var varierte, da studentene ble påvirket på forskjellige måter. S4-A3 sa \emph{"jeg begynner jo å lure på hva som skjer, og pasienten blir kanskje enda mer urolig, og redd for at jeg skal gå."} S3-A3, som skulle utføre sårskift, påpekte at hun ble stresset, mens S5-A3 ikke ble plaget. S9-A3 kommenterte at hun kunne tenke seg en bekreftelse på at noen faktisk har gått inn på rommet hvor signalet ble utløst fra.

\noindent
Under steg fire, i scenarioet med sårskift, så S3-A3 på telefonen før hun valgte å avvise signalet. S2-A2 kommenterte etterpå at hun heller ville sett på rompanelet hvor det ringte fra, og latt telefonen ringe ut, \emph{... og da hadde det vært fint om det ikke var lyd på telefonen.} 
Flere påpekte også at lyden fra telefonen kan være forstyrrende for pasientene. 

\noindent
S9-A3: \emph{"det er jo noen som blir skremt, men andre sier "blir du ringt på nå.." liksom. For de hører en lyd fra lommen og mange synes det er helt greit om vi går."}

\noindent
S6-A4: \emph{"de føler de må forte seg å snakke ferdig fordi du har det sikkert travelt eller sånn."}

\noindent
S2-A2 mente det er feil at pasientene må vise forståelse og godta at andre signaler skal forstyrre dem. Hun mente det er viktig at sykepleierene tar ansvar for at slike situasjoner blir håndtert på en god måte slik at pasientene ikke føler at de er til bry.

\subsubsection{Andre funn}
Det var tydelig at studentenes bruk og forståelse av systemet ikke var helt i samsvar med tenkt bruk. S1-A1 fortalte at på hennes avdeling ble telefonen kun brukt til å ringe med, og   ikke til å motta pasientsignaler. Hun visste derimot ikke hvorfor den ikke ble brukt.
Det var også noe uklart hvem som ble oppringt ved et pasientsignal, da S3-A3 trodde primær og disponibel sykepleier mottok pasientsignalet samtidig. 
Flere studenter fortalte at de kan glemme å markere seg som tilstede, men at sykepleierene er flinkere til dette. 

\subsection{Funn etter introduksjon av prototype}
Under gjennomgangen av prototypens funksjonalitet uttrykte flere av studentene at de syntes valgene vi hadde gjort var gode. Likevel fikk vi mange gode innspill, og studentene diskuterte potensielle utfordringer og kom med forslag til annen funksjonalitet. 

\subsubsection{Endring som følge av ny funksjonalitet}
Det var av spesielt stor nytte under workshopene å se hvordan bruk av prototypen endret hvordan scenarioene utspilte seg. S1-A1 og S6-A4, som begge skulle utføre sårskift, benyttet seg av muligheten til å sette seg som utilgjengelige før de gikk inn til pasienten. S6-A4 valgte i tillegg å si ifra til sykepleieren på tunet om at hun ville være opptatt en stund. Da S2-A2, som skulle inn til en urolig pasient, ble spurt om hun ville sagt ifra til noen før hun gikk inn, svarte hun \emph{"hvis appen skal funke, burde man vel ikke trenge det."}

\noindent
Det var mye diskusjon angående varsling av innkommende pasientsignaler, og hvordan denne burde tilpasses de ulike statusene. Flere uttrykte at ved status som opptatt eller utilgjengelig, kunne vibrasjon eller lavere lyd være et godt alternativ. Ingen ville derimot sette seg som utilgjengelig hvis det innebar å ikke motta noen form for varsling.

S1-A1: \emph{"... ja, det er alltid for og imot det å få jobbe uforstyrret, men kan miste annen informasjon."}

\noindent
S2-A2 påpekte risikoen ved at alle sykepleierene på et tun setter seg som utilgjengelige.  
Studentene diskuterte hvordan sykepleierenes status vil påvirke hvem som mottar pasientsignaler, og det kom flere forslag til hvordan dette kunne løses. Blant annet ble det foreslått at pasientsignaler først blir sendt til de som er tilgjengelige, deretter de som er opptatte, og eventuelt til slutt til de som er utilgjengelige. S2-A2 foreslo at de som er opptatt eller utilgjengelig først burde varsles med kun vibrasjon. Dersom signalet ble sendt til alle sykepleierene uten at det ble godtatt, burde derimot alle varsles med lyd.  
For å forhindre at man blir stående som utilgjengelig når man ikke er det, ble det foreslått at systemet automatisk endrer status til tilgjengelig når man forlater rommet, at telefonen gir en påminnelse om at man fremdeles står som utilgjengelig etter en viss tid, eller at man har en tidsbegrensing på hvor lenge man kan stå som utilgjengelig. 

\noindent
Flere foreslo også å flytte rompanelene nærmere pasienten, da det ble påpekt at de står så langt unna at det kan være vanskelig å se hva som står. Ofte vil det være forstyrrende å snu seg, eller flytte seg, for å se hva som står. S3-A3 foreslo å flytte rompanelet til veggen bak sykesengen \emph{"... så du slipper å følge med bak. Og så ser ikke pasienten det selv, så vi kan følge med hvis det er noe."} Spesielt hvis man ikke skulle hatt lyd ville det være enklere å oppdage innkommende pasientsignaler.

\noindent
S5-A3 foreslo å legge til Felleskatalogen, da sykepleierene ofte bruker sin personlige mobiltelefon til å slå opp i denne. Hun påpekte samtidig at hun ikke ville lagt til mer funksjonalitet, \emph{...så vi holder på kontakten med pasienten og ikke blir gående med nesen i telefonen."}

