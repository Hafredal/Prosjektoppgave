\section{Workshopene}
\label{ws}
I denne delen vil vi legge frem funn vi gjorde i løpet av WS1 og WS2. Datamaterialet ble analysert ved bruk av stegvis-deduktiv induktiv metode, og resultatene som presenteres er de av relevans for våre forskningsspørsmål.

\subsection{Oversikt}
Det var ønskelig å avklare i hvilken grad sykepleierne har oversikt over inneliggende pasienter på sitt sengetun, og over kollegers ansvar og aktiviteter.

\noindent
Under vaktskiftemøtet blir det gitt detaljert informasjon om alle pasientene på sengetunet, og sykepleierene fordeler ansvaret for disse. Flere av deltagerne forteller at de har med seg en skriftlig oversikt over pasientene, hvor detaljer fra vaktskiftemøtet noteres. S1-A1 fortalte at hun bruker denne listen hvis hun skal inn til en pasient hun ikke har vært hos før. Denne oversikten opprettholdes med mindre uforutsette hendelser oppstår, eksempelvis at pasienter bytter rom eller det kommer noen som må ha øyeblikkelig hjelp. Det kom også frem at det etterstrebes at en pasient har den samme primærsykepleieren over tid så godt det lar seg gjøre.

\noindent
S1-A1 poengterte at selv om sykepleierene har oversikt over kollegers ansvar kan de ikke vite hvor alle er til en hver tid. Derimot vil kunnskapen om kollegenes ansvar gi en pekepinn på hvor det er sannsynlig å finne dem. I tillegg er det også utbredt bruk av telefon for å få tak i hverandre. 

\noindent
Hvorvidt det er vanlig å si ifra til kolleger når man gå noe annet sted, for eksempel inn til en pasient, til et annet tun, avdeling eller etasje, eller til lunch viste seg å være svært variabelt. Noen ville sagt ifra om de gikk inne til en pasient og ble borte en stund, samtidig som andre ikke ville gjort det da de regnet med at kolleger vet hva som skal gjøres på det gitte rommet og at det kan ta lengre tid.
Dette var også tydelig under utspillingen av scenarioene uten prototype, hvor kun én av fire varslet sykepleieren tilstede på tunet om at hun skulle inn på et pasientrom. På spørsmål om man ville varslet andre før man går inn på et rom med en urolig pasient (noe som kan ta lengre tid) svarte S5-A3: \emph{"ja, man ville jo egentlig sagt det, men det er jo sjelden det blir sagt da, for de andre er jo gjerne klare over hva som skjer."}. Dette vitner om at sykepleierene regner med at alle har oversikt, både over hvilke pasienter som ligger på hvilke rom og hvem som har ansvar for de forskjellige pasientene.

\noindent
Etter introduksjon av prototypen, og muligheten til å sette sin status til  utilgjengelig for å formidle nettopp dette til kolleger så vi at dette var en funksjon som ble brukt i større grad enn å gi muntlig beskjed.
S1-A1 og S6-A4, som begge skulle utføre sårskift, benyttet seg av muligheten til å sette seg som utilgjengelig før de gikk inn til pasienten. S6-A4 valgte i tillegg å si ifra til sykepleieren på tunet om at hun ville være opptatt en stund. Da S2-A2, som skulle inn til en urolig pasient, ble spurt om hun ville sagt ifra til noen før hun gikk inn, svarte hun at: \emph{"hvis appen skal funke, burde man vel ikke trenge det"}.

\noindent
For å forhindre at man blir stående som utilgjengelig når man ikke er det, noe som ville virke mot hensiktem med å gi sykepleierene bedre oversikt over kolleger, ble det foreslått forskjellige løsninger. (1) At systemet automatisk endrer status til tilgjengelig når man forlater rommet, (2) at telefonen gir en påminnelse om at man fremdeles står som utilgjengelig etter en viss tid, (3) at man har en tidsbegrensing på hvor lenge man kan stå som utilgjengelig.


\subsection{Håndtering av innkommende pasientsignal}
Under workshopene kom det frem at sykepleierene tar hensyn til flere faktorer når de vurderer og håndterer innkommende pasientsignaler. 

\noindent
Alle deltagerene var enige om at dersom det var en pasient de selv hadde primæransvar for ville de, såfremt de hadde mulighet, gå til pasienten med en gang.
Flere av studentene fortalte også at de ved anrop fra en pasient de ikke er primær for, lar telefonen ringe en stund for å se om noen andre (hovedsaklig primærsykepleier) svarer. Dette til hensyn for pasienten, slik at denne skal slippe å forholde seg til flere sykepleiere enn nødvendig. I tilfeller hvor pasientsignalet ikke ble besvart ville noen av deltagerene gått til inn til pasienten, mens andre først ville hørt med eventuelle kolleger i nærheten om det ikke var noen av dem som har primæransvar for den aktuelle pasienten. Som ble poengtert av S1-A1 ville det ikke lenger være noen vits i fordelingen av pasienter mellom sykepleierene dersom alle svarer på alle anrop.

\noindent
En annen faktor som ble vektlagt var kunnskapen sykepleierene har om tilstanden til pasienten som har utløst signalet, og at dersom denne var dårlig ville de i større grad prioritere å se til pasienten fremfor å holde fast ved ansvarsfordelingen. Samtidig påpeker studentene at man aldri vet hvorfor pasienten utløser signalet, og at det kan være alt fra at pasienten vil ha et glass vann, må på do, eller at det er noe alvorlig. Det eneste de vet er at rommet trenger assistanse.  
Dersom sykepeieren befant seg sammen med en annen pasient når signalet ble utløst ville også tilstanden til denne tas i betraktning. 

\noindent
For å være i bedre stand til å prioritere hvordan et pasientsignal skal behandles av den enkelte sykepleier, var det stor oppslutning om forslaget å få et oversiktsbilde av rommet hvor alarmen er utløst fra på telefonen, slik at man får et inntrykk av pasientens tilstand, og hvor i rommet pasienten befinner seg. S4-A3 påpekte at det kan være nyttig å få et enkelt punkt eller en diagnose, som kort forklarer hva som feiler pasienten. På denne måten slipper man å se på listen man har med seg. Dette forslaget kom etter at vår prototype var introdusert som mulig løsning. Deltagerene kom også med forslag om å oppgi hvor i rommet signalet er utløst fra (ved sengen, ved døren, på toalettet osv). 

\noindent
For å gi sykepleierene mer informasjon om hva signalet gjelder ble det foreslått å ha forskjellige signalknapper for pasientenes ulike behov. Det ble lagt frem forslag om egne signalknapper for toalett, mat eller drikke, eller annen assistanse. Deltagerene kom imidlertid frem til at selv om dette i mange tilfeller kunne gi god støtte for prioritering dersom man allerede var opptatt, var det kanskje ikke en ønsket funksjon av flere grunner. For det første kan noen av pasientene som ligger inne være svært syke, og kan få problemer med å velge riktig knapp. Dersom en pasient skulle få et illebefinnende og den eneste knappen vedkommende nådde var signalet for drikke, kunne dette i verste fall medføre fatale følger. Studentene var usikre på om mindre akutte signaler, eksempelvis ønske om drikke, kunne bli nedprioritert av sykepleierene, og at pasientene ville måtte vente lenge på hjelp. De var også redde for at pasientene ville få en høyere terskel for å tilkalle hjelp, hvis de "bare vil ha et glass vann" for eksempel.

\subsection{Varsling fra telefon og veggpanel}
Det var interessant å finne ut hvorvidt det innkommende pasientsignalet påvirker sykepleiernes arbeid, og i så fall hvordan. Det kom tydelig frem under utspillingen av scenarioene at dette varierte veldig. Noen uttrykte at de ble stresset, mens andre sa at det ikke plaget dem. S4-A3 tenkte mest på hvordan det påvirket pasienten hun var hos, mens S9-A3 savnet å få en bekreftelse på at noen faktisk hadde gått inn til pasienten som utløste signalet slik at hun kunne slutte å tenke på det. 

\noindent
Det ble ved flere anledninger under WS1 og WS2 påpekt at det i visse situasjoner ville være positivt om telefonen ikke ringte like høyt, eller bare vibrerte. Dette var i stor grad av hensyn til pasientene, da flere av studentene opplever at pasienter blir skremt av at det ringer, eller føler seg til bry og forter seg å snakke ferdig. 
S2-A2 mente det er feil at pasientene må vise forståelse og godta at andre signaler skal forstyrre situasjonen. Hun mente det er viktig at sykepleierne tar ansvar for at dette blir håndtert på en god måte slik at pasientene ikke føler at de er til bry. Det ble foreslått at telefonene ikke skulle gi noen form for varsling. Under scenarioet med sårskift valgte S1-A1 å sette seg som utilgjengelig. Hun fikk derfor beskjed av fasilitator om at nå har det gått en alarm på et annet rom som du ikke blir varslet om. Dette førte til at deltagerne under fokusgruppediskusjonen i etterkant i stor grad ble enige om at dette var lite hensikstmessig, da de faktisk ønsket å ha en viss oversikt over hva som skjedde. I forbindelse med denne diskusjonen kom det også forslag om å flytte veggpanelet fra døren til veggen over sengen til pasienten. Dette gjaldt spesielt dersom signalet skulle endres til å bli varslet med svakere, eller ingen lyd, slik at man likevel ikke går glipp av inkommende pasientsignaler.

\noindent
Etter introduksjon av prototypen ble det diskutert hvordan sykepleiernes status vil påvirke hvem som mottar pasientsignaler, og det kom flere forslag til hvordan dette kan løses. Det ble blant annet foreslått at signalet først blir sendt til de som er tilgjengelige, deretter de som er opptatte, og til slutt til de som er utilgjengelige. S2-A2 foreslo at de som er opptatte eller utilgjengelige først burde varsles med kun vibrasjon. Dersom signalet ble sendt til alle sykepleierne uten at noen gikk inn til pasienten, burde derimot alle varsles med lyd. S1-A1 påpekte at hun så absolutt ville ha tydelig varsling dersom en av pasientene hun hadde ansvar for hadde utløst et signal som hadde gått lenge uten å bli besvart. Hun mente det var viktig å være sikker på at pasienten får hjelp innen rimelig tid. 