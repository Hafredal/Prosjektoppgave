\section{Workshopene}
\label{ws}


\subsection{Dagens system}

\subsubsection{Oversikt over pasienter}
Det var innledningsvis ønskelig å avklare i hvilken grad sykepleierene har oversikt over inneliggende pasienter på sitt sengetung. Studentene forteller at ved vaktskiftemøte blir informasjon om pasientene gitt til de som går på vakt.

\noindent
S3: \emph{"Det er ganske nøye på rapportene, så du vet hvem alle er og hva det dreier seg om, forandringer fra dag til dag, selv om du ikke har vært inne hos dem."}

\noindent
Flere av deltagerene forteller at de har en liste over inneliggende pasienter, hvor detaljer fra vaktskiftemøtet noteres.

\noindent
S1: \emph{"Ja så hvis man skal inn til en pasient, hvis jeg ikke har vært der før, så tar jeg opp lappen."}

\noindent
S1 forteller videre at denne oversikten opprettholdes, med mindre uforutsette hendelser oppstår, eksempelvis at pasienter bytter rom eller det kommer noen som må ha øyeblikkelig hjelp.
 
 
\subsubsection{Oversikt over kolleger og varsling}
Sykepleierenes oversikt over kollegers deres ansvar og aktiviteter, er av stor relevans til vår oppgave, og vi ønsket derfor å undersøke om sykepleierene informerer hverandre om sine egne aktiviteter.

\noindent
S1: \emph{"Man kan vel ikke vite hvor alle er hele tiden, men man vet hvor de har ansvar. Så hvis man skal ha tak i noen, ringer man på telefonen, eller ser at vedkommende har ansvar for rom 1 og 2, så da sjekker man der. Men vi har ikke noe peiler på hvor alle er til enhver tid nei."} 

\noindent
Under fokusgruppediskusjonen, kom det tydelig frem at dette varierer i stor grad.

\noindent
J: \emph{"Men sier dere ifra når man går noe sted."}

\noindent
S1: \emph{"Hvis man bare skal opp i fjerdeetasje eller andreetasje for å hente noe, tror jeg de bare går. Det er det inntrykket jeg har fra min avdeling hvertfall."} 

\noindent
S3: \emph{"De sier jo i fra."}

\noindent
S4: \emph{"Ikke hele tiden men, litt begge deler."}

\noindent
Dette var også tydelig under utspillingen av scenarioene uten prototype, hvor kun én av fire varslet sykepleieren tilstede på tunet om at hun skulle inn på et pasientrom. 

\subsubsection{Håndtering av innkommende pasientsignal}
Det kom tydelig frem under workshopene at sykepleierene tar hensyn til flere faktorer når de vurderer og håndterer innkommende pasientsignaler. 
\begin{itemize}
\item Flere av studentene fortalte at de ved anrop fra en pasient de ikke er primær for, lar telefonen ringe en stund for å se om andre svarer. Dette til hensyn for pasienten, slik at denne skal slippe å forholde seg til flere sykepleiere, og for å holde en oversikt, som S1 kommenterte: \emph{"ellers kan jo alle bare gå inn overalt, og da er det ikke noe vits i den fordelingen av pasienter lenger."} Hvis derimot et signal ikke besvares, poengterte S6: \emph{"hvis det ringer på et rom og jeg spør rundt meg "er det noen av dere som har 203?" og ingen reagerer, så går jeg - folk kan jo være på toalettet eller noe"}  
\item Under utspillingen av scenario uten prototype, hvor sykepleieren er inne hos Jenny og det utløses et pasientsignal fra Adams rom, sa S4: \emph{"først ville jeg avvist signalet for å se om noen andre ville tatt den, når jeg først står her inne og snakker."} Da de andre deltagerne ble spurt om hvordan de ville håndert en slik situasjon, svarte S2: \emph{"du har jo litt overblikk over hva som feiler de andre pasientene, om noen ha veldig dårlig pust eller veldig dårlig rytme, og du ser at den ringer på, så må du gå og sjekke."}  
\item På spørsmål om sykepleieren ville hatt mer informasjon om pasienten som har utløst signalet, svarer S7: \emph{"vet ikke, det er jo alltid noen rom som ringer ofte, og om ting som ikke haster, men man skal jo egentlig gå å se. Man kan jo ikke regne med at han bare skal ha et glass vann."}
\end{itemize}

LYDER/LYS/OSV/FORSTYRRELSE

\subsubsection{Varsling fra telefon}
For hvert utspilte scenario ønsket vi å finne ut hvordan det innkommende pasientsignalet påvirket sykepleierens arbeid. Funnene var varierte, da studentene ble påvirket på forskjellige måter. S4 sa \emph{"jeg begynner jo å lure på hva som skjer, og pasienten blir kanskje enda mer urolig, og redd for at jeg skal gå."} S2, som skulle utføre sårskift, påpekte at hun ble stresset, mens S5 derimot, ikke ble plaget.

\subsubsection{Varsling fra veggpanel}
Hvorvidt lyd og varsling fra panelene er forstyrrende eller ikke, er det delte meninger om.  

ANDRE TING
- vet ikke hvordan systemet fungerer

\subsubsection{Oppsummering}