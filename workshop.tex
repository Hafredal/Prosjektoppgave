\section{Workshopene}
\label{ws}

I denne delen vil vi legge frem funn vi gjorde i løpet av de to worshopene som ble holdt.  Workshopene ble filmet, noe som har gitt oss muligheten til å gå tilbake å se på det utolkede materialet. Vi ønsket å finne ut av hvordan dagens system fungerer, hva som er positivt og hvilke utfordringer studentene opplever. Vi fant at det er behov for endringer av systemet, både med tanke på hvilken informasjon som formidles ved utløst signal, og hvordan dette formidles.

\subsection{Oversikt og awareness}
Det var ønskelig å avklare i hvilken grad sykepleierene har oversikt over inneliggende pasienter på sitt sengetun, og over kollegers ansvar og aktiviteter.

\noindent
Studentene fortalte at informasjon om pasientene og fordelig av ansvar blir nøye gjennomgått for de som går på vakt under vaktskiftemøtet, og at det er dette som er gunnlaget for oversikten gjennom vakten. Rapporteringen er nøyaktig nok til at du har oversikt over hvem alle pasientene er og endringer i deres tilstand fra dag til dag selv om du ikke har hovedansvar for gitte pasient. Det kom også frem at det etterstrebes at en pasient har samme primærsykepleier så godt det lar seg gjøre. Flere av deltagerene forteller at de har med seg en skriftlig oversikt over inneliggende pasienter, hvor detaljer fra vaktskiftemøtet noteres, og at denne kan brukes til oppslag gjennom dagen. Denne oversikten opprettholdes med mindre uforutsette hendelser oppstår, eksempelvis at pasienter bytter rom eller det kommer noen som må ha øyeblikkelig hjelp. 

\noindent
S1-A1 poengterer at selv om sykepleierene har oversikt over kollegers ansvar kan de ikke vite hvor alle er til en hver tid. Derimot vil kunnskapen om kollegenes ansvar gi en pekepinn på hvor det er sannsynelig å finne dem. I tillegg er det også utbredt bruk av telefon for å få tak i hverandre. 

\noindent
Når det kom til hvorvidt det var vanlig å si ifra til kolleger når man gå noe annet sted, for eksempel inn til en pasient, til et annet tun, avdeling eller etasje, eller til lunch viste det seg at dette var svært variabelt. Noen ville sagt ifra om de gikk inne til en pasient og ble borte en stund, samtidig som andre ikke ville gjort det da de regnet med at kolleger vet hva som skal gjøres på det gitte rommet og at det kan ta lengre tid.

Dette var også tydelig under utspillingen av scenarioene uten prototype, hvor kun én av fire varslet sykepleieren tilstede på tunet om at hun skulle inn på et pasientrom. På spørsmål om man ville varslet andre før man går inn på et rom med en urolig pasient (noe som kan ta lengre tid) svarte S5-A3: \emph{"ja, man ville jo egentlig sagt det, men det er jo sjelden det blir sagt da, for de andre er jo gjerne klare over hva som skjer."}. Dette vitner om at sykepleierene regner med at alle har oversikt, både over hvilke pasienter som ligger på hvilke rom og hvem som har ansvar for de foskjellige pasientene.
