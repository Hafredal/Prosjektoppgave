\section{Resultatenes kvalitet}
\label{chp: analyse}

Vi vil her se på forhold som kan ha påvirket resultatene som er beskrevet tidligere i kapittelet. Vi vil holde oss til det som har hatt direkte innvirkning på resultatene her, og komme med en analyse av andre faktorer i [REFERERE].

\subsection{Pålitelighet}
Idealet innenfor en positivistisk tradisjon er en nøytral eller objektiv observatør, og forskerens engasjement vil da kunne betraktes som støy som påvirker resultatene. Man har i senere tid innsett at fullstendig nøytralitet ikke kan eksistere og at forskerens engasjement i mange tilfeller kan sees som en ressurs. Det er da viktig at det blir eksplisitt redgjort for hvordan dette brukes i en analyse, og hvordan det kan ha preget forskningsarbeidet i form av utvalg (blant annet av deltagere, sitater, teorier), datagenerering, analyse og resultater. Bruk av diktafon eller videopptak gjør at forskeren kan legge frem sistater direkte slik intervjuobjektet la det frem, noe som kan øke undersøkelsens pålitelighet. 

\subsubsection{Rollen som moderator}
Vi stilte selv som moderatorer under workshopenes fokusgruppediskusjoner, noe som var en ny erfaring for begge. At vi selv tok rollen som moderator ga oss god kontroll over hvilken retning samtalen tok under fokusgruppediskusjonene, gjennom kontroll over spørsmålene som ble stilt og når diskusjonen ble avbrutt. På denne måten kunne vi sikre at diskusjonene holdt seg til temaer vi var interessert i. På en annen side er farene ved å styre diskusjonen i for stor grad er at man risikerer å gå glipp av interessant informasjon som kunne kommet frem dersom diskusjonen ikke ble stoppet, og at man vet å stille for detaljerte og gejntagende spørsmål til en viss grad kan legge ord i munnen på dem. Da ingen av oss hadde hatt en slik rolle tidligere var det vanskelig å gjøre en god vurdering på hvorvidt vi klarte dette på en god måte underveis i workshopen. Vi så på videoopptakene i ettertid at selv om dette ble gjort noen grad,  var diskusjonene bra og informative, og det kom mange gode innspill.

\subsubsection{Rollen som fasilitator}
Den som hadde moderatorrollen hadde også rollen som fasilitator under utspilling av scenarioene. Dette var  i likhet med moderatorrollen en ny erfaring, noe som ga samme manglende grunnlag til å evalurere gjennomføringen underveis. Dersom spørsmål stilles på en ugusntig måte kan dette påvirke handlingene til deltagerene, samt lede dem til å forklare fremgangsmåter slik de er ment å skulle gjøres fremfor slik de blir gjort i praksis. Gjennom videoopptakene ser det imidertid ut til at deltagerene ikke lar seg påvirke på denne måten.

\subsubsection{Deltagere på workshop}
I samarbeid med vår veileder kom vi frem til at vi trengte mellom tre og fem deltagere på hver workshop. Da vi hadde noe problemer med å få nok deltagere til å melde seg på hadde vi ikke muligheten til å selv sette sammen grupper utifra ulike kriterier blandt de som hadde meldt interesse. Eksempler på slike kriterier kunne vært om de hadde brukt telefonen under praksis, alder, kjønn og type avdelinger. Fordelen med å kunne gjøre utvelging basert på slike kriterier er at vi kunne satt sammen grupper hvor det hadde vært grunn til å forvente et godt samspill og gode diskusjoner. Dette betyr ikke at vi gruppene eller diskusjonene på noen måte var dårlige, men at de kunne blitt ennå bedre dersom vi hadde hatt muligheten til å velge.

\subsection{Gyldighet}
Gyldighet knyttes til spørsmålet om resultatene vi kommer frem til faktisk svarer på forskningsspørsmålene vi har stilt. Dette kan styrkes gjennom at vi er åpne om hvordan forskningen er blitt gjennomført, og begrunnelser for de valgene som er tatt med tanke på metoder for datagenerering og teoretiske innspill til analysen. 

\subsubsection{Deltagerenes erfaringer}
Alle deltagerene var også sykepleierstudenter, noe som betyr at de kun hadde noen uker med erfaring, i tillegg var det ikke alle deltagerene som hadde brukt telefonen. Vi antar at dette påvirker resultatene da det ble en del synsing fra deltagerene og referering til hva de trodde de "vanlige" sykepleierene bruker å gjøre i forskjellige situasjoner.  

\subsubsection{Kunstig situasjon}
Da workshopene ble holdt på NSEPs brukbarhetslab ble dette et nytt miljø for deltagerene. På tross av at laboratoriet har flere rom, sykesenger, legefrakker og pasienttøy, samt våre mock-ups av forskjellige veggpaneler vil det nye miljøet påvirke deltagerene i en viss grad. Spesielt med tanke på handlinger som er blitt en vane i det vante miljøet som kan føre til usikkerhet i det ukjente. Den kunstig situasjonen kan også gjøre deltagerene mer bevisst på hva de gjør og dermed føre til at en ikke får et sant bilde av hva som er faktisk praksis. 

\subsubsection{Prototypen}
Prototypen som ble brukt i de to siste scenarioene på de to dagene var allerede laget av oss. Denne var ikke fullt operativ med tanke på kommunikasjon med andre enheter (kunne for eksempel ikke motta pasientsignalene) noe som ble poengtert under forklaringen av prototypen. Den kan likevel ha fremstått som mer ferdig enn den var og dermed påvirket tilbakemeldigene fra deltagerene, blandt annet i form av at den hindret deltagerenes kreativitet, eller at de ikke ønsket å kritisere detaljer av hensyn til oss.

\subsubsection{Problemer underveis}
Under den andre workshopen oppsto det missforståelser under utspilling av det første scenarioet der deltageren som hadde rollen som sykepleier trodde signalet som kom inn på telefonen var et hasteanrop og ikke et vanlig pasientsignal. Dette førte til at valgene som ble tatt var anderledes enn det som ellers ville blitt gjort. Det var interessant å se hvordan deltageren reagerte i en slik situasjon, og det ga perspektiv til reaksjonene når delatagerene visste at det var et vanlig pasientsignal. Likevel var dette utenfor hva vi var ute etter i utgangspunktet, men da dette gjeldt kun én av totalt åtte scenarioer som ble utspilt ser vi ikke på det som avgjørende for resultatene vi fikk. 

\subsection{Generalisering}
Det har i lenger tid pågått diskusjon om hvorvidt generalisering er nødvendig i kalitativ forskning, og i så fall hvordan dette skal gjøres. Generaisering beskriver i hvor stor grad resultatene er gyldige i andre situasjoner enn den som er studert. 
