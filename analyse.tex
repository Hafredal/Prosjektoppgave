\section{Resultatenes kvalitet}
\label{chp:analyse}

Vi valgte workshop som metode for datagenerering da vi ønsket å avdekke brukerenes erfaringer og vurdering av det systemet de bruker i dag, samtidig som vi ønsket å teste vår foreslåtte løsning. Da vi i dette prosjektet har hatt en begrensning på tid, var det ønskelig å generere data på en effektiv og produktiv måte \cite{Tjora}. Vi vil her vurdere forhold som kan ha påvirket datamaterialets kvalitet.

\subsection{Pålitelighet}
Hva angår materialets pålitelighet, vil vi her forsøke å redegjøre for de interne forholdene i forskningen som kan ha påvirket resultatene.

\subsubsection{Rollen som moderator}
Vi stilte selv som moderatorer under workshopenes fokusgruppediskusjoner, og dette var en ny erfaring for oss begge. At vi selv tok denne rollen ga oss god kontroll over hvilken retning samtalen tok, da vi kontrollerte hvilke spørsmål som ble stilt og når diskusjonen skulle avbrytes. Dette førte til at diskusjonen fokuserte på de temaene som var av relevans til våre forskningsspørsmål. På en annen side kan dette ha styrt diskusjonen i for stor grad, og det er en risiko for at interessant informasjon kunne kommet frem dersom diskusjonen ikke ble stoppet. Ved å stille detaljerte og gjentagende spørsmål kan man til en viss grad ha lagt ord i munnen på deltagerene. Det var vanskelig å gjøre en god vurdering av hvorvidt vi fylte denne rollen på en god måte underveis i workshopene. I ettertid kunne videoopptakene avdekke at det til tider ble stilt spørsmål som kan ha vært ledende, men inntrykket var likevel at informantene ikke lot seg påvirke av dette i vesentlig stor grad da de kom med flere innspill og nye ideer som vi som forskere ikke hadde vurdert selv.

\subsubsection{Rollen som fasilitator}
Den som hadde moderatorrollen hadde også rollen som fasilitator under utspillingen av scenarioene. Dette var i likhet med moderatorrollen en ny erfaring, og det var vanskelig å evaluere gjennomføringen underveis. Da vi hadde rollen som både fasilitator og forsker, var det utfordrende å opprettholde fullstendig nøytralitet, som påpekt av både Svanæs og Seland (2014) og Tjora (2012). Spørsmål kan derfor ha blitt stilt på en slik måte at det påvirket deltagerenes handlinger. Selv med videoopptak var det i ettertid vanskelig å avgjøre i hvor stor grad dette skjedde. 

\subsubsection{Deltagere på workshop}
I samråd med vår veileder kom vi frem til at vi trengte mellom tre og fem deltagere til hver workshop. Da vi hadde problemer med å få nok deltagere til å melde seg på, var vi i en situasjon hvor vi ikke kunne plukke ut deltagere for å få en representativ gruppesammensetning \cite{Seland, Cavaye95}. Som Berg (1999) beskriver, vil en av utfordringene med CSCW-systemer være at slike systemer brukes av et bredt spekter av brukere. I 2006 hadde sykehuset 3263 ansatte sykepleiere \cite{nokkeltall}, og hvis vi antar at dette antallet ikke har blitt betraktelig mye lavere, kan man spørre seg hvorvidt ni deltagere kan representere et så stort antall, og om det i det hele tatt er mulig å inkludere alle i en slik prosess \cite{Cavaye95}. Ved å arrangere to uavhengige, men like, workshoper kunne vi sammenligne funnene for å se om de varierte avhengig av individene som deltok. Hovedtrekkene av resultatene var i stor grad like, selv om detaljer ble vurdert forskjellig av deltagerene. Dette var derimot individuelle forskjeller begge dager, og vi kan derfor ikke trekke store skilnader mellom WS1 og WS2. I slike grupper er det en risiko for at noen individers meninger kommer tydeligere frem enn andres, og hvorvidt deltagerene faktisk var enige eller uenige var vanskelig å avgjøre. Alle deltagerene, med ett unntak, var kvinnelige studenter i samme aldersgruppe som oss, men det er usikkert hvorvidt denne relasjonen hadde påvirkning på resultatene.

\subsection{Gyldighet}
Gyldighet knyttes til spørsmålet om resultatene vi kommer frem til faktisk svarer på forskningsspørsmålene vi har stilt. Dette kan styrkes ved at vi er åpne om hvordan forskningen er blitt gjennomført, og begrunnelser for de valgene som er tatt med tanke på metoder for datagenerering og teoretiske innspill til analysen. 

\subsubsection{Deltagernes erfaringer}
Alle deltagerene var studenter, og hadde dermed begrenset erfaring med bruk av systemet vi ønsket å teste. Likevel har de, som Svanæs og Seland (2004) anbefaler, direkte erfaring med den type arbeid som ble utforsket, og siden de er potensielle fremtidige brukere av pasientsignalsystemet anså vi deres deltagelse som nyttig. For å kunne evaluere utfordringer og fordeler med det eksisterende systemet og dagens arbeidspraksis, var det interessant å se hvordan bruk og rutiner varierte. Deltagerene hadde spesielt variert erfaring med bruk av telefonen for mottak av pasientsignal. Dette førte til at en del av evalueringen ble synsing, og ikke direkte avledet fra eget bruk. Flere av studentene refererte i blant til hva de trodde de "vanlige" sykepleierene ville gjort, eller vanligvis gjør, i ulike situasjoner. 

\subsubsection{Kunstig situasjon}
Som påpekt av Alsos og Dahl (2008) er det viktig at de fysiske testomgivelsene, scenarioene og prototypene er så realistiske som mulig for å kunne generere gyldige resultater. Samtidig påpeker Berg (1999) at det for CSCW-systemer kan være vanskelig å gjenskape konteksten hvor systemet skal implementeres i et laboratorium. Det at workshopene ble holdt på NSEPs brukbarhetslab, et nytt og annerledes miljø for deltagerne, kan dermed ha påvirket workshopens realisme. Til tross for at laboratoriet har flere rom, sykesenger, legefrakker og pasienttøy, samt våre mock-ups av forskjellige veggpaneler, kan det nye miljøet ha påvirket deltagerne. Den kunstige situasjonen kan ha gjort deltagerne mer bevisste på sine handlinger, og resultatene vil ikke nødvendigvis gjenspeile hva som er vanlig praksis. Det kan likevel hevdes å være en trygghet at alle deltagerene fikk observere de utspilte scenarioene, og kunne diskutere situasjonene i ettertid, og eventuelt kommentere hva man ville ha gjort annerledes. Det er naturligvis en risiko for at det man sier man ville gjort ikke nødvendigvis gjøres i praksis, men totalt sett vil en slik felles diskusjon kunne avdekke reell arbeidspraksis i større grad. Deltagerne ble i etterkant av begge workshoper spurt hvorvidt de mente scenarioene var realistiske og relevante, og alle deltagere ga uttrykk for at de var det. Selv om scenarioene ga et større inntrykk av sykepleiernes handlinger i ulike situasjoner, vil man ikke få et komplett bilde, da det er umulig å gjenskape alle mulige situasjoner. Ved alle scenarioene som ble utspilt, spilte en av deltagerne en sykepleier som var tilgjengelig på tunet, og vi fikk dermed ikke undersøkt om sykepleieren ville håndtert situasjonen annerledes hvis dette ikke var tilfellet.

\subsubsection{Prototype og mock-ups}
Til tross for at vi poengterte at prototypen bare var skjermbilder med noe interaksjon, og ikke en operativ løsning, kan den ha fremstått som mer ferdig enn det den var, og dermed påvirket tilbakemeldigene fra deltagerene. Det kan ha hindret deltagernes kreativitet, eller frahindret dem fra å komme med kritikk av hensyn til oss. Ifølge Mogensen og Trigg (1992) må de kontekstuelle artefaktene gjenspeile en viss grad av realisme. Det at det eksisterende systemet ble forsøkt etterlignet med mock-ups, og at varslingene avvek fra virkeligheten, kan ha påvirket workshopenes realisme. 

\subsubsection{Problemer underveis}
Under WS2 oppsto det misforståelser under utspillingen av det første scenarioet, da deltageren som hadde rollen som sykepleier trodde det innkommende signalet på telefonen var et hasteanrop og ikke et vanlig pasientsignal. Dette førte til at valgene hun tok var annerledes enn det hun ellers ville gjort, noe som kunne ha gitt feilaktige resultater hadde det ikke blitt avdekket. Vi kan derimot ikke garantere at andre misforståelser ikke kan ha oppstått.

\subsection{Generalisering}
Generalisering beskriver i hvor stor grad resultatene er gyldige i andre situasjoner enn den som er studert. Vi vil derfor argumentere for at (1) funnene er relevante for sykepleierstudentenes avdelinger, og (2) at de er relevante for lignende arbeidsmiljø. 
