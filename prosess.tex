\section{Prosess}
\label{chp: prosess}

Prosessen i et forskningsprosjekt kan beskrives som sekvensen av aktiviteter som utføres i løpet av prosjektets varighet \cite{Oates}. Vi vil nå gjøre rede for de metodene og tilnærmingene vi har valgt i vår prosess.

\subsection{Litteraturstudie og analyse av tidligere arbeid}
Teorien vi har lagt frem i kapittel 2 er basert på et litteraturstudie gjort innledningsvis av arbeidet. For å definere forskningsspørsmålene knyttet til oppgaven, analyserte vi i første omgang tidligere arbeid gjort av Evjemo, Klemets og Kristiansen. Dette ga oss en følelse av forskningsområdet og utfordringene knyttet til det eksisterende pasientsignalsystemet. 
De temaene som utpekte seg som spesielt interessante var awareness, avbrudd og medvirk
	
\subsection{Strategi}
Dette er den helhetlige tilnærmingen for å svare på forskningsspørsmålet. 

\subsection{Prototype}


\subsection{Paradigme}
Bruk Tjora og Oates - SDI
Kvalitativ VS kvantitativ

\subsection{Workshop}
Tjora (2012) påpeker at valg av metode for datagenerering må reflektere hva man faktisk ønsker å finne ut, og at effektivitet bør vektlegges, “datagenereringen må kunne frambringe mest mulig relevant og pålitelig informasjon uten unødig bruk av forskeres og deltakeres tid og ressurser”.

\subsubsection{Deltakere}
\subsubsection{Forberedelse}
\subsubsection{Utførelse}

\subsection{Dataanalyse}

\subsubsection{Datamaterialets kvalitet}


