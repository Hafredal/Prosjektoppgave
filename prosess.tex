\section{Prosess}
\label{chp: prosess}

Prosessen i et forskningsprosjekt kan beskrives som sekvensen av aktiviteter som utføres i løpet av prosjektets varighet. Vi vil nå gjøre rede for de metodene og tilnærmingene vi har valgt i vår prosess.

\subsection{Dokumentstudie og analyse av tidligere arbeid}
	Først for å få en generell forståelse, deretter mer konkret
	
\subsection{Strategi}
Dette er den helhetlige tilnærmingen for å svare på forskningsspørsmålet. Som strategi har vi valgt design and creation, da vi ønsker å designe og teste en protoype av pasientsignalsystemet. Dette er en typisk problemløsende tilnærming, og er normalt en iterativ prosess som består av fem steg; bevissthet, forslag, utvikling, evaluering og konklusjon. 

	Bevissthet og forslag et resultat av litteraturen
	Utvikling - prototype
	Evalurering og konklusjon er gitt av de neste punktene - generering og analyse

\subsection{Metode for datagenerering}
Tjora (2012) påpeker at valg av metode må reflektere hva man faktisk ønsker å finne ut, og at effektivitet bør vektlegges, “datagenereringen må kunne frambringe mest mulig relevant og pålitelig informasjon uten unødig bruk av forskeres og deltakeres tid og ressurser”.
WORKSHOP

\subsection{Deltakere}

\subsection{Dataanalyse}

\subsection{Paradigme}
Bruk Tjora og Oates - SDI
Kvalitativ VS kvantitativ

\subsection{Datamaterialets kvalitet}


