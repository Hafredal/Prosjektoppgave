\subsection{Suksesskriterier for implementering av informasjonsystemer ved sykehus}
\label{chp: suksess_sykehus}

Vi har i dette kapittelet sett på noen av elementene som kan påvirket graden av suksess ved implementering av nye informasjonsystemer. Mange av dem, som brukbarhet og motstand [EDIT ETTER BEHOV], er generelle for utvikling og implementering og vil ikke bli gått nermere inn på i denne delen. Vi vil heller trekke frem spesielle hensyn som er viktige å ta når man utvikler og implementerer informasjonsystemer spesielt for sykehus og det spesielt komplekse systemet dette er.



\noindent
Slike strømlinjeformede og rasjonaliserte informasjonsystemer vil også føre til et stort behov for workarounds, nettopp fordi det hele tiden vil oppstå situasjoner hvor det ikke er ønskelig, eller mulig å gjennomføre en oppgave slik informsjonsystemet tilsier at det skal gjøres. 

\noindent
Vi ser gjennom dette at informasjonsystemer utviklet for bruk på sykehus må ha en svæt stor åpenhet, og så liten grad av fastsatte rutiner og sekvenser som mulig.