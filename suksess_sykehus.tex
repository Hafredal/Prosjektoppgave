\subsection{Suksesskriterier for implementering av informasjonsystemer ved sykehus}
\label{chp: suksess_sykehus}

Vi har i dette kapittelet sett på noen av elementene som kan påvirket graden av suksess ved implementering av nye informasjonsystemer. Mange av dem er generelle, som brukbarhet og motstand [EDIT ETTER BEHOV], for utvikling og implementering og vil ikke bli gått nermere inn på i denne delen. Vi vil heller trekke frem spesielle hensyn som er viktige å ta når man utvikler og implementerer informasjonsystemer spesielt for sykehus og det spesielt komplekse systemet dette er.

\noindent
Det å utvikle en CSCW-applikasjon helseomsorgen vil aldri bli en enkel prosess. Det er en tydelig konflikt mellom det flytende samarbeidet og tilsynelatende men nødvendig rotete arbeismåten til sykepleiere og den formelle, standardiserte og relativt stive funksjonaliteten til et informasjonssystem. Derfor er en av forutsettningene for et suksessfullt system i et slikt miljø å ikke forsøke å erstatte denne 'rotetheten' med en strømlinjeformet og rasjonalitet som ofte er vanlig for slike systemer. Verktøy som kun har forutbestemte sekvensiell trinn, eller som tillater kun gitte typer data-input vil derfor ikke fungere sammen med måten sykepleierene arbeider på, og som en følge av dette ikke overleve.\cite{Berg99}

\noindent
Slike strømlinjeformede og rasjonaliserte informasjonsystemer vil også føre til et stort behov for workarounds, nettopp fordi det hele tiden vil oppstå situasjoner hvor det ikke passer, eller ikke er mulig å gjennomføre en oppgave slik informsjonsystemet tilsier at det skal gjøres.

\noindent
Vi ser gjennom dette at informasjonsystemer utviklet for bruk på sykehus må ha en svæt stor åpenhet, og så liten grad av fastsatte rutiner og sekvenser som mulig.