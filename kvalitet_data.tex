\subsection{Kvalitet av data}
\label{kvalitet av data}

Det brukes tre kriterier for kvaliteten av kalitative studier \cite{Tjora}. Disse er pålitelighet (intern logikk gjennom forskningsprosjektet), gyldighet (logisk sammenheng mellom prosjektets utforming og funn og og de spørsmål en søker svar på) og generaliserbarhet (forskningens gyldighetsområde utover det som faktisk er undersøkt). Tjora (2012) legger selv til de to kriteriene transparens og refleksivitet. 

\subsubsection{Pålitelighet}
Idealet innenfor en positivistisk tradisjon er en nøytral eller objektiv observatør, og forskerens engasjement vil da kunne betraktes som støy som påvirker resultatene. Man har i senere tid innsett at fullstendig nøytralitet ikke kan eksistere og at forskerens engasjement i mange tilfeller kan sees som en ressurs. Det er da viktig at det blir eksplisitt redgjort for hvordan dette brukes i en analyse, og hvordan det kan ha preget forskningsarbeidet i form av utvalg (blant annet av deltagere, sitater, teorier), datagenerering, analyse og resultater. Bruk av diktafon eller videopptak gjør at forskeren kan legge frem sistater direkte slik intervjuobjektet la det frem, noe som kan øke undersøkelsens pålitelighet. 

\subsubsection{Gyldighet}
Gyldighet knyttes til spørsmålet om resultatene vi kommer frem til faktisk svarer på forskningsspørsmålene vi har stilt. Dette kan styrkes gjennom at vi er åpne om hvordan forskningen er blitt gjennomført, og begrunnelser for de valgene som er tatt med tanke på metoder for datagenerering og teoretiske innspill til analysen. 

\subsubsection{Generalisering}
Det har i lenger tid pågått diskusjon om hvorvidt generalisering er nødvendig i kalitativ forskning, og i så fall hvordan dette skal gjøres. Generaisering beskriver i hvor stor grad resultatene er gyldige i andre situasjoner enn den som er studert. Tre former for generalisering i kalitativ forskning er skissert under.

\begin{itemize}
\item Naturalistisk generalisering: redegjøre så godt for detaljene i det som er forsket på at leseren selv kan vurdere gyldighet, for eksempel i forhold til egen forskning
\item Moderat generalisering: forskeren selv forklarer i hvilke situasjoner resultateen vil være gyldige
\item Konseptuell generalisering: det utvikles konsepter som har gyldighet utover det som er studert
\end{itemize}

\subsubsection{Transparens}
Transparens er knyttet til presentasjonen av forskningen. Blandt annet hvordan udnersøkelser er gjort, problemer som har oppstått, på hvilke tidspunt valg ble tatt og hvilke teorier som er benyttet er spørsmål som bør besvares av forskeren. Hansikten er at leseren skal kunne selv vurdere kvaliteten på forskningen som er gjennomført. 

\subsubsection{Refleksivitet}
Tolkning av empirisk data må følges av en refleksjon over hvordan tolkningen kommer frem, eller en tolkning av vår egen tolkning. Det må tas i betraktning hvordan tolkningene av data blir påvirket av forskerens egne erfaringer, teoretiske, sprøklige, politiske og kulturelle omgivelser og muligheter. Ved å være klar over dise og gjøre slik refleksjon eksplisitt øker forskningens troverdighet. 

