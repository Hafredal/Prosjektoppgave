\subsection{Workarounds}
\label{chp: workarounds}

Workarounds ("midlertidige løsninger"\ på norsk) defineres av Kobayashi(2005) som \emph{"informal temporary practices for handling exceptions to normal workflow"}, og er en vanlig teknikk for å håndtere den iboende usikkerheten i dynamiske arbeidsmiljø, som et sykehus må sies å være. Selv om slike workarounds er vanlige betyr ikke dette at de alltid er utelukkende suksessfulle. En workaround som gir organisatoriske løsninger for unntak som stadig gjentar seg, og dermed reduserer den kognitive innsatsen som kreves for å håndtere nye krisesituasjoner kan kalles suksessfulle. Dersom workarounden derimot skaper ringvirkninger av ustabilitet i resten av organisasjonen, ser vi på dette som en lite suksessfull workaround, selv om den løser problement der og da. \cite{Kobayashi05}

\noindent
Et svært viktig kritere for å kunne ta riktige avgjørelser på hva som vil være en god workaround i en krisesituasjon er god awareness. Dette vil gi besluttningstakeren et godt grunnlag for å kunne avgjøre om workarounden vil være suksessfull eller ikke. Dersom mye informasjon gis muntlig de ansatte mellom, vil dette være til hinder for gode avgjørelser da det er større fare for at beslutninger tas på grunnlag av ufullstendig informasjon. Et annet hinder for gode workarounds er for spesifiserte roller og arbeidssekvenser. Dette kan føre til at det ikke finnes tilgjengelig personell når en krisesituasjon oppstår, og man blir nødt til å velge et dårligere alternativ. Til sist viser empiriske studeier at sykepleiere, i en problemløsnings-situasjon, ofte kun ser på det akutte problemet fremfor hva som er den opprinnleige åsraken. Dette fører til problemer med å opprette varige løsninger på problemet, noe som fører til både at problemene vil fortsette å oppstå og at men ikke ser hvilke ringvirkninger ens beluttninger kan føre med seg, og dermed kan gi dårlige workarounds. \cite{Kobayashi05}

\noindent
Et eksempel på en workaround finner vi i \cite{KlemetsRedundancy}, hvor de ser på bruken av trådløse telefoner i forbindelse med pasientalarmer. Sykepleierene mottar pasientsignalene til pasientene de er ansvarlige for direkte på den overnevnte telefonen, og kan på denne måten få beskjed om at pasienten trenger hjelp uansett hvor de er. Det viser seg i midlertid at denne telefonen blir brukt i svært liten grad på noen avdelinger, som blandt annet infeksjon. Her tar ikke sykepleierene med seg telefonen inn til pasientene, noe de begrunner med at den likevel ikke kan benyttes når de har på seg smittevernsfrakk. \cite{KlemetsRedundancy}

\noindent
Workarounds kan være nødvendig når det oppstår akutte situasjoner, men kan også oppstå som følge av sperrer i systemet som er implementert. Dette kan være sperrer som er designet med en hensikt, eller sperrer som følger av dårlig design. Et eksempel på førstnevnte finner vi i \cite{Vogelsmeier08}, hvor sykepleierene ikke kunne bestille doser av medisisner høyere enn det som er anbefalt. I tilfeller hvor høyere doser likevel var skrevet ut av legen måtte pleierene da bestille flere doser på en gang.  

\noindent
På en annen side, dersom workarounds oppstår som konsekvens av utilsiktede sperrer, eller der systemet er for rigid i forhold til sykepleierenes arbeidsmønster, slik at systemet ikke støtter opp om arbeidet på en tilfredstillende måte, er dette svært uheldig. Dette kan i verste fall føre til livstruende situasjoner.