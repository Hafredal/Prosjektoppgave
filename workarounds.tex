\subsection{Workarounds}
\label{chp: workarounds}

Workarounds defineres av Kobayashi (2005) som \emph{"informal temporary practices for handling exceptions to normal workflow"}. Direkte oversatt til norsk betyr det "å jobbe rundt", eller å finne midlertidige løsninger.
\noindent
Workarounds kan være nødvendig når det oppstår akutte situasjoner hvor man ikke har nødvendige ressurser tilgjengelig, eller de kan oppstå som følge av sperrer i et system. Disse sperrene kan være tilsiktede, eller utilsiktede. Et eksempel på førstnevnte finner vi i \cite{Vogelsmeier08}, hvor sykepleiere ikke kan bestille doser av medisiner høyere enn det som er anbefalt. I tilfeller hvor høyere doser likevel var skrevet ut av lege, bestilte pleierene bare flere doser. 
Et annet eksempel på en workaround er gitt av Klemets, Evjemo og Kristiansen (2013), som beskriver hvordan sykepleiere fordeler ansvar for pasienter når de skal gå til lunsj. I utgangspunktet fordeles ansvar for pasienter gjennom en bemanningsplan som konfigureres i en applikasjon kjørende på en PC i sengeområdet. Å endre på denne planen krever merarbeid, og noen sykepleiere velger derfor å ikke gjøre endringene, men heller gi muntlig beskjed til kolleger om at en går til lunsj. Dette fører til at telefonene ringer under lunsjpausen.

\noindent
Vogelsmeier (2008) beskriver workarounds som førstegrads problemløsing i den forstand at man lager mekanismer for å jobbe rundt problemer, uten å forsøke å løse den underliggende årsaken til at problemet oppsto.
Dersom workarounds oppstår som konsekvens av utilsiktede sperrer, eller der systemet er for rigid i forhold til sykepleierenes arbeidsmønster slik at systemet ikke støtter opp om arbeidet på en tilfredstillende måte, er dette svært uheldig. Dette kan i verste fall føre til livstruende situasjoner.
Selv om slike workarounds er vanlige i medisinske settinger, er de som beskrevet ikke nødvendigvis effektive og vellykkede. Workarounds som gir organisatoriske løsninger for unntak som stadig gjentar seg, og dermed reduserer den kognitive innsatsen som kreves for å håndtere nye krisesituasjoner, vil ofte være suksessfulle. Workarounds som derimot gir ringvirkninger av ustabilitet i resten av organisasjonen, kan sies å være lite suksessfulle, selv om de løser problemet der og da \cite{Kobayashi05}.