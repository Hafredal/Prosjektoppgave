\section{Suksesskriterier ved implementering av informasjonsystemer}
\label{chp: suksesskriterier}

Om et informasjonsystem er suksessfullt eller ikke, og dermed også dets skjebne avgjøres sammen av de ansatte, mellomledelsen og toppledelsen der systemet er implementert. Vi vil presisere at  vi med suksessfullt her mener at det er utbredt i bruk, og at brukertilfredsheten er høy. Det er ikke mulig å definere et sett med faktorer som smmen gir en oppskrift på suksess, selv når det er full enighet om målene for implementasjonen \cite{Berg01}.

\noindent
Enkel tilgang til teknologi for kommunikasjon med andre sykepleiere, og sanntidsinformasjon om kollegers tilgjengelighet, oppgaver og pasientstatus, er viktig for å støtte god arbeidsflyt \cite{Ebright10}. 
Det finnes  mange fallgruver ved utvikling og implementering av nye applikasjoner, og vi beskriver her et utvalg av disse, basert på relevans til vår oppgave.


