\section{Suksesskriterier ved implementering av informasjonsystemer}
\label{chp: suksesskriterier}

Om et informsjonsystem er suksessfullt eller ikke, og dermeds også dets skjebne avgjøres sammen av de ansatte, mellomledelsen og toppledensen \cite{Berg01}. Vi vil presisere at når vi med suksessfullt her mener at det er utbredt i bruk, og at brukertilfredsheten er høy. Det er ikke mulig å detfinere et sett med faktorer som smmen gir en oppskrift på suksess, selv ikke når det er full enighet om målene for implementasjonen \cite{Berg01}.

\noindent
Enkel tilgang til teknologi for kommunikasjon med andre sykepleiere, og veldesignede informasjonstavler som gir sanntidsinformasjon om kollegers tilgjengelighet, oppgaver og pasientstatus, er viktig for å støtte god arbeidsflyt \cite{Ebright10}. 
Det finnes  mange fallgruver ved utvikling og implementering av nye applikasjoner, og vi beskriver her et utvalg av disse, basert på relevans til vår oppgave.


