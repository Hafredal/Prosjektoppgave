\section{Suksesskriterier ved implementering}
\label{chp: suksesskriterier}


\noindent
Enkel tilgang til teknologi for kommunikasjon med andre sykepleiere og godt designede informasjonstavler, som gir sanntidsinformasjon angående kollegers tilgjengelighet, oppgaver og pasientstatus, er viktig for å støtte opp om god arbeidsflyt\cite{Ebright10}.


foreløpig struktur:


- MMI (mmi-boken)

Desinere av grensesnitt har gjennom årene kommet frem til en rekke reningslinjer for god design av skjermbiler. Disse er ofte blitt kritisert for å være både for spesifikke og ufullstendige\cite{mmi}, defor må designere også jobbe med tanke på, og en forståelse for hva slags informasjon brukerene trenger, og hvordan denne skal presenteres\cite{Ebright10}. Vi kommer likevel ikke utenom at design av skjermbildet er en nøkkelkomponent for vellykkede interaktive systemer\cite{mmi}. Seks kategorier av prinsipper presentert i \cite{mmi} er (1) Eleganse og enkelhet,  (2) Målestokk, kontrast og proposjoner, (3) Organisasjon ov visuell struktur, (4) Modul og program, (5) Bilder og representasjon og (6) stil.

\noindent
\emph{Gestaltprinsippene}

\noindent
Gestaltprinsippene har sitt navn fra det tyske Gestalten, som betyr "å forme". Prinsippene blir ofte referert til som lover, og det finnes mange varianter utviklet av forskjellige psykologer, men de har til felles at de forklarer hvordan vi organiserer visuelle intrykk i områder og strukturer. Disse ble i utgangspunktet brukt til å foreslå hvordan statiske visuelle elementer burde presenteres for effektive resultater\cite{Chang02}. Chang, Dooly og Tuovien (2002) identifiserer og presenterer elleve prinsipper med betydelig relevans for design av skermbilder. 

\noindent
Prinsippet om \emph{Balanse og symetri} sier at visuelle objekter som ikke er balansert vil det synes ufullstendig.

\noindent
Prinsippet om \emph{forlengelse} omhandler hvorden øyet instingtivt følger linjer i synsfeltet

\noindent
Prinsippet om \emph{lukkethet} peker på at vi ubevisst prøver å lukke ufullstendige former for at de skal gi mening

\noindent
Prinsippet om \emph{figur - bakgrunn}: vi skiller automatisk mellom forgrunn og bakgrunn. endringer i forgrunn- og bekgrunnfarge kan føre til at vi ser forskjellige figurer, selv om bildet er det samme.

\noindent
Prinippet om \emph{Fokuspunkt}: Fokuspunktet er det punktet vi legger merke til først, og dette vil være vårt utgangspunkt for hvordan, og i hvilken rekkefølge vi ser resten av bildet. Som eksempel vil et element som distinkt skiller seg fra resten fange vår oppmerksomhet først.

\noindent
Prinsippet om \emph{Isomorft samsvar}:Vi tolker bildet med utgangspunkt i våre erfaringer. Derfor er det ikke gitt at et bilde eller symbol vi gi samme mening for alle.

\noindent
Prisnippet om \emph{God form}: vi vil gjøre en så god figur ut fra det vi ser som mulig. En god form er et enkelt design eller en symetrisk utforming.

\noindent
Prinsippet om \emph{Nærhet}: vi vil oppfatte objekter som er plassert nær hverandre, adskilt fra andre som en gruppe, og anta at de er relatert til hverandre på noen måte.

\noindent
Prinsippet om \emph{Likhet}: Lignende objekter vil ha samme funksjon som objekter plassert nær hverandre, og vi vil se dem som en gruppe, med antagels om realsjon.

\noindent
Prinsippet om \emph{Enkelhet}: Underbevisstheten vår prøver å forenkle det vi ser til noe vi kjenner igjen. Kompleks grafikk med mye unødvendige elementer kan føre til at vi trekker utilsiktede konklusjoner.

\noindent
Prinsippet om \emph{Harmoni}: Impliserer at det finnes en kongruans mellom elementer i et design. De kan se ut som om de hører sammen som om det er en visuell knytning mellom dem som gjør at de kommer sammen.


- hvorfor cscw applikasjoner failer (Grudin88)

- workarounds (Kobayashi10)

- suksesskriterier generelt (se endringsledelseboken)

- suksesskriterier sykehus (se Ebright10)

\noindent



