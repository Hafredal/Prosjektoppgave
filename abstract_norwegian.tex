\pagestyle{empty}
\renewcommand{\abstractname}{Sammendrag}
\begin{abstract}
\noindent
Å opprettholde awareness om kollegers arbeid er avgjørende for god koordinasjon. Sykepleiere mottar informasjon nødvendig for å opprettholde slik awareness gjennom eksterne avbrudd. Disse avbruddene kommer i tillegg til det allerede avbruddsdrevende miljøet de arbeider i.
Denne prosjektoppgaven ser på pasientsignalsystemet ved St.Olavs Hospital i Trondheim. Vi har sett på hvordan dette fungerer i dag, hvordan sykepleierene opprettholder awareenss på arbeidsplassen og hvorden det avbruddsdrevede miljøet påvirker deres arbeid. 

\noindent
Gjennom studier av tidligere arbeid har vi forsøt å identifisere forbedringspotensialer ved systemet som benyttes i dag, og på bakgrunn av dette laget en prototype med forslag til endringer. Denne ble testet gjennom workshops. Vi presenterer resultatene fra disse og dikuterer hvorvidt informasjon om kollergers aktiviteter er nyttig, hvordan denne bør distribueres og hvordan systemet kan endres for å begrense negative effekter av avbrytelser.

\end{abstract}