\pagestyle{empty}
\renewcommand{\abstractname}{Sammendrag}
\begin{abstract}
\noindent
Et sykehus er en organisasjon som krever effektivitet og kommunikasjon for å sikre god pasientomsorg og sikkerhet. For sykepleiere kan dermed informasjon om kollegers aktiviteter  være avgjørende for koordinasjon og prioritering av arbeid. Tidligere forskning antyder at slik informasjon ofte formidles muntlig, men at det kan oppstå situasjoner hvor denne type kommunikasjon kan være utfordrende. Informasjonsflyt og eksterne avbrytelser fører til at sykepleiere kontinuerlig må omprioritere sine aktiviteter.

\noindent 
Denne prosjektoppgaven omhandler pasientsignalsystemet ved St. Olavs Hospital i Trondheim. Vi har utforsket dagens arbeidspraksis for å prøve å avdekke hvorvidt informasjon om kollegers aktiviteter og tilgjengelighet er nyttig for sykepleiere i sin prioritering og organisering av arbeid, og hvordan slik informasjon kan distribueres på en hensiktsmessig måte. Med bakgrunn i teori om at eksterne avbrytelser kan påvirke individers kognitive kapasitet, og potensielt medføre negative effekter, har vi vurdert hvorvidt det er mulig å redusere slike effekter.
Gjennom studier av tidligere arbeid har vi forsøkt å identifisere fordeler og utfordringer ved systemet som benyttes, og på bakgrunn av dette laget en prototype med forslag til forbedringer. Med workshop som metodisk tilnærming for å besvare forskningsspørsmålene, ble prototypen testet og evaluert av sykepleierstudenter. Vi vil i denne oppgaven presentere resultatene fra disse, og diskutere hvorvidt informasjon om kollegers aktiviteter og tilgjengelighet er nyttig, og hvordan slik informasjon bør distribueres.

\end{abstract}